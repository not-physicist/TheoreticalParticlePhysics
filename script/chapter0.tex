\setcounter{chapter}{-1}
\chapter{Organisational}
Tutorials:
Thursday:   8-10, 10-12
Friday:     10-12, 13-15 

\paragraph{Exam} consists of four problems
\begin{itemize}
   \item first quickies
   \item 2nd-4th: similar in style to homework; two will be very close to homework 
\end{itemize}
One needs $50\%$ of points from homework. May hand in pairs.

\paragraph{Content of the lectures}
\begin{itemize}
   \item \sm of particle physics
   \item Electroweak sector
      \begin{itemize}
         \item gauge principle
         \item Higgs mechanism
         \item Yukawa interactions
         \item CP-violation
      \end{itemize}
\end{itemize}

\paragraph{Exercises}
\begin{itemize}
   \item go through basics of computing Feynman diagrams
   \item not to derive the formalism
   \item Lagrange $\rightarrow$ Feynman rules $\rightarrow$ amplitudes $\rightarrow$ cross section and decay rates (measured quantities)
\end{itemize}

\paragraph{Literature}
\begin{itemize}
   \item Halzen and Martin, Quarks and Leptons (a lot of basics of QCD)
   \item Cheng and Li (includes also quantum field theory topics CP-violation in \sm)
   \item Mark Thomson
   \item QFT basics
      \begin{itemize}
         \item Peskin and Schroeder
         \item M.Schwartz
         \item Ryder
      \end{itemize}
   \item Okun, Leptons and Quarks
\end{itemize}
