\chapter{Lorentz Transformation}
\section{Introduction}
Metric (used for distance measuring)
\begin{align}
   g_{\mu\nu} = \begin{pmatrix} 1 & 0 & 0 & 0 \\ 0 & -1 & 0 & 0 \\ 0 & 0 & -1 & 0 \\ 0 & 0 & 0 & -1\end{pmatrix}
\end{align}
In string and general relativity people tend to use $\diag(-, +, +, +)$.

\begin{align}
   p &= (E, \pmb{p}) \\
   p^2 &= E^2 - \pmb{p}^2 = m^2
\end{align}

Light is always light-like
\begin{align*}
   t^2 - (x^2 + y^2 + z^2) = 0
\end{align*}

Greek indices always go from $0$ to $3$
\begin{align}
   r^2 = g_{\mu\nu} r^\mu r^\nu = t^2 - \pmb{r}^2
\end{align}

Distance between two spacetime point is defined via
\begin{align}
   \left| r_A - r_B \right|  = \sqrt{(r_A - r_B)\cdot(r_A - r_B)} = \sqrt{r_A^2 + r_B^2 - 2r_A\cdot r_B}
\end{align}

\section{Lorentz Transformation}
\textit{Lorentz Transformation} is transformation between two inertial frames moving with constant velocity $\pmb{v}$ with respect to each other (boosts).
\begin{align*}
   x &= (x_0, x_1, x_2, x_3) \\
   x' &= (x'_0, x'_1, x'_2, x'_3) \\
   x_0 &= ct = t ;\quad c=1
\end{align*}

We define
\begin{align}
   \beta = \frac{v}{c} \quad \text{or} \quad
   \pmb{\beta} = \frac{\pmb{v}}{c}
\end{align}

Coordinates in these two frames are related like
\begin{align*}
   x'_0 &= \gamma(x_0 - \beta x_1) \\
   x'_1 &= \gamma (x_1 - \beta x_0) \notag \\
   x'_2 &= x_2 \notag \\
   x'_3 &= x_3 \notag \label{math:lorTrafo}
\end{align*}

Inverse transformation with $\beta \mapsto -\beta$

$\gamma$-factor is defined via
\begin{align}
   \gamma = \frac{1}{\sqrt{1-\pmb{\beta}^2}}
\end{align}
Since $|\pmb{\beta}| \leq 1 \Rightarrow \gamma \geq 1$

Alternative parametrization
\begin{align*}
   \beta &= \tanh(\zeta), \quad \gamma = \cosh(\zeta) \\
   \gamma \beta &= \sinh(\zeta)
\end{align*}

Insert this into equation (\ref{math:lorTrafo})
\begin{align*}
   x'_0 = x_0 \cosh(\zeta) - x_1 \sinh(\zeta) \\
   x'_1 = x_0 \sinh(\zeta) - x_1 \cosh(\zeta)
\end{align*}

We can turn this into matrices
\begin{align}
   \begin{pmatrix} x_0 \\ x_1 \\ x_2 \\ x_3\end{pmatrix}
   &=
   \begin{pmatrix} \gamma & \gamma\beta & 0 & 0 \\ \gamma\beta & \gamma & 0 & 0 \\ 0 & 0 & 1 & 0 \\ 0 & 0 & 0 & 1\end{pmatrix}
   \cdot
   \begin{pmatrix} x'_0\\ x'_1 \\ x'_2 \\ x'_3\end{pmatrix} \notag \\
   x &= \Lambda x'
\end{align}

\section{Mathematical Properties of Lorentz Transformation}

Distance is invariant under Lorentz transformation
\begin{align}
   s^2 = x_0^2 - x_1^2 + x_2^2 + x_3^2 = x^2
\end{align}

Lorentz transformation includes
\begin{itemize}
   \item Rotation and boosts
   \item Parity $\pmb{x} \mapsto -\pmb{x}$
   \item Time reversal $t \mapsto -t$
\end{itemize}

We can also expand it with translation. It then turns to Poincare group.

\subsection{Tensors}
Define a function of original coordinates ($\alpha=0,1,2,3$)
\begin{align}
   x'^\alpha = x'^\alpha (x^0, x^1, x^2, x^3)
\end{align}

If $x'^\alpha$ transforms like
\begin{align}
   x'^\alpha = \frac{x'^\alpha}{x^\beta} x^\beta
\end{align}
it is called \textit{contravariant}

Consider derivative $\frac{\partial}{\partial x'^\alpha}$
\begin{align}
   \frac{\partial f(x)}{\partial x'^\alpha} = \frac{\partial f(x)}{\partial x^\beta} \frac{\partial x^\beta}{\partial x'^\alpha}
\end{align}
We can see the $x'$ is now in the denominator. The objects transformed like this are called \textit{covariant}.

Consider the following generic objects: $A'^\alpha$ contravariant vector
\begin{align}
   A'^\alpha = \frac{\partial x'^\alpha}{\partial x^\beta} A^\beta
\end{align}

$B'_\alpha$ is covariant
\begin{align}
   B'_\alpha = \frac{\partial x^\beta}{\partial x'^\alpha} B_\beta
\end{align}
Note $(x^0, x^1, x^2, x^3)$ is contravariant.

The field strength tensor $F'^{\alpha\beta} = \frac{\partial x'^\alpha}{\partial x^\gamma} \frac{\partial x'^\beta}{\partial x^\delta} F^{\gamma\delta}$ is contravariant rank 2.

Mixed is also allowed $H'^\alpha_\beta = \frac{\partial x'^\alpha}{\partial x^\gamma} \frac{x^\delta}{\partial x'^\beta} H^\delta_\gamma$ 

\paragraph{Inner or scalar product}
\begin{align*}
   B' \cdot A' &= B'_\alpha A'^\alpha \\
               &= \left( \frac{\partial x^\beta}{\partial x'^\alpha} B_\beta \right) \left( \frac{\partial x'^\alpha}{\partial x^\gamma} A^\gamma \right) \\
               &= \frac{\partial x^\beta}{\partial x^\gamma} B_\beta A^\gamma \\
               &= \delta^\beta_\gamma B_\beta A^\gamma = B \cdot A
\end{align*}

\begin{align}
   \dd{s}^2 &= \left( \dd{x^0} \right)^2 - \left( \dd{\pmb{x}} \right)^2 \notag \\
            &= (g_{\alpha\beta} \dd{x^\alpha}) \dd{x^\beta} = \dd{x_\beta} \dd{x^{\beta}}
\end{align}
Thus we can use metric tensor to lower index $\dd{x_\beta} = g_{\alpha\beta} \dd{x^{\alpha}}$

\begin{align*}
   A^\alpha = \left(A^0, \pmb{A} \right) \\
   A_\alpha = \left(A^0, -\pmb{A} \right)
\end{align*}

\section{Matrix Representation of Lorentz Transformation}
\subsection{General Properties}
We have 
\begin{align*}
   x = \begin{pmatrix} x^0 \\ x^1 \\ x^2 \\ x^3 \end{pmatrix} \quad gx =  \begin{pmatrix} x^0 \\ -x^1 \\ -x^2 \\ -x^3 \end{pmatrix} 
\end{align*}

Then $a\cdot b = (a, gb) = g_{\mu\nu} a^\mu b^\nu = (ga, b) = a^Tgb = (ga)^T b$

\begin{align}
   x'^\mu = \Lambda^\mu_{\;\nu} x^\nu \mapsto x' = \Lambda x \\
   x \cdot x = x' \cdot x' = (\Lambda x) (\Lambda x)
\end{align}

\begin{align*}
   g_{\mu\nu} x^\mu x^\nu &= g_{\sigma \tau} x'^\sigma x'^\tau \\
                          &= g_{\sigma \tau} \Lambda^\sigma_{\; \mu} x^\mu \Lambda^\tau_{\;\nu} x^\nu \\
                          &= g_{\sigma \tau} \Lambda^\sigma_{\; \mu} \Lambda^\tau_{\;\nu}x^\mu  x^\nu 
\end{align*}

Then we have the \textit{defining} rule of Lorentz group
\begin{align}
   g_{\mu\nu} &= g_{\sigma \tau}  \Lambda^\sigma_{\; \mu} \Lambda^\tau_{\;\nu} \\
   g &= \Lambda^T g \Lambda
\end{align}

\paragraph{Properties}
\begin{itemize}
   \item $\left|\det(\Lambda) \right| = 1$
   \item $\left|\Lambda^0_{\;0} \right| \geq 1$
\end{itemize}

The orthochronous Lorentz transformations $\Lambda$ forms a group. \\

Parity does not form a group
      \begin{align}
         \Lambda_P = \text{diag}(1,-1,-1,-1)
      \end{align}

Time reversal 
\begin{align}
   \Lambda_T = \text{diag}(-1,+1,+1,+1)
\end{align}

There are four classes of Lorentz transformations depending on $\left(\sgn(\det(\Lambda)), \sgn(\Lambda^0_0) \right)$
\begin{itemize}
   \item $(+, +)$ $\Lambda$
   \item $(- ,-)$ $\Lambda_T \Lambda$ 
   \item $(-, +)$ $\Lambda_P \Lambda$
   \item $(+, -)$ $\Lambda_T\Lambda_P \Lambda$
\end{itemize}

Orthochronous $\Lambda$ has 6 parameters, $3$ for boosts and $3$ for rotations. $\Lambda^T g \Lambda = g$ is actually $16$ equations. All matrices here are symmetric. Thus $6$ of $16$ are redundant. There are $10$ independent equations.
$\Lambda$ has $16$ entries and it has $16-10=6$ free parameters.

\subsection{Explicit Construction}
We will restrict ourselves in orthochronous Lorentz transformations. The exponential function is defines via Taylor expansion. With $L \in \R^{4 \times 4}$
\begin{align*}
   \Lambda = \euler^L = \exp(L)
\end{align*}

From linear algebra we know
\begin{align}
   \det(\Lambda) = \det(\euler^L) = \euler^{\tr(L)}
\end{align}

Since $\det(\Lambda) = 1$, $\tr(L) = 0$

\begin{align*}
   \Lambda^T g \Lambda &= g \\
   g \Lambda^T g \Lambda &= \id_4 \\
   g \Lambda^T g &= \Lambda^{-1} \\
   \exp(g L^T g) &= \Lambda^{-1} = \exp(-L) \\
   \Leftrightarrow g L^T g &= -L \\
   \Leftrightarrow (gL)^T &= - gL
\end{align*}

This means that $gL$ is anti-symmetric
\begin{align*}
   L = \begin{pmatrix} 0 & L_{01} & L_{02} & L_{03} \\ L_{01} & 0 & L_{12} & L_{13} \\ L_{02} & L_{12} & 0 & L_{23} \\ L_{03} & L_{13} & L_{23} & 0\end{pmatrix}
\end{align*}

Define six basis matrices $S_{1,2,3}$ and $K_{1,2,3}$
\begin{align*}
   &S_1 = \begin{pmatrix} 0&0&0&0 \\ 0&0&0&0 \\ 0&0&0&-1 \\ 0&0&1&0\end{pmatrix} \quad 
   &&S_2 = \begin{pmatrix} 0&0&0&0 \\ 0&0&0&1 \\ 0&0&0&0 \\ 0&-1&0&0\end{pmatrix} \quad
   &&&S_3 = \begin{pmatrix} 0&0&0&0 \\ 0&0&-1&0 \\ 0&1&0&0 \\ 0&0&0&0\end{pmatrix} \\
   &K_1 = \begin{pmatrix} 0&1&0&0 \\ 1&0&0&0 \\ 0&0&0&0 \\ 0&0&0&0\end{pmatrix} \quad 
   &&K_2 = \begin{pmatrix} 0&0&1&0 \\ 0&0&0&0 \\ 1&0&0&0 \\ 0&0&0&0\end{pmatrix} \quad 
   &&&K_3 = \begin{pmatrix} 0&0&0&1 \\ 0&0&0&0 \\ 0&0&0&0 \\ 1&0&0&0\end{pmatrix} \quad 
\end{align*}
%TODO: all matrices

$S_i$ is the generator of $3$-dimensional rotations and $K_i$ is the generator of $3$-dimensional boosts.
\begin{align*}
   \hat{n} \in \R^3, & \quad |\hat{n}| = 1\\
   \hat{n} \cdot \pmb{S} &= n_1S_1 + n_2 S_2 + n_3 S_3 \\
   (\hat{n}\cdot \pmb{S})^3 &= - \hat{n} \cdot \pmb{S} \\
   (\hat{n}\cdot \pmb{K})^3 &= + \hat{n} \cdot \pmb{S}
\end{align*}

In the end
\begin{align}
   L &= -\pmb{\omega} \cdot \pmb{S} - \pmb{\zeta}\cdot \pmb{K} \quad \text{with }\pmb{\omega}, \pmb{\zeta} \in \R^3 \\
   \Lambda &= \exp(-\pmb{\omega} \cdot \pmb{S} - \pmb{\zeta}\cdot \pmb{K})
\end{align}

We now will look at concrete examples
\begin{itemize}
   \item $\pmb{\zeta} = 0,\; \pmb{\omega} = w \hat{e}_z$
      \begin{align*}
         \Lambda = \begin{pmatrix} 1 & 0 & 0 & 0 \\ 0 & \cos{\omega} & \sin{\omega} & 0 \\ 0 & -\sin{\omega} & \cos{\omega} & 0 \\ 0 & 0 & 0 & 0\end{pmatrix}
      \end{align*}
      Rotational angle is $\omega$.
   \item $\pmb{\omega} = 0, \; \pmb{\zeta} = \zeta \hat{e}_x$
      \begin{align*}
         \Lambda = \begin{pmatrix} \cosh{\zeta} & -\sinh{\zeta} & 0 & 0 \\ -\sinh{\zeta} & \cosh{\zeta} & 0 & 0 \\ 0 & 0 & 1 & 0 \\ 0 & 0 & 0 & 1\end{pmatrix}
      \end{align*}
\end{itemize}

% Now we want to calculate the algebra $ \left[ S_i, K_j \right] $ and $ \left[ S_i, S_j \right]  $
