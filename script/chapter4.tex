%%%%%%%%%%%%%%%%%%%%%%%%%%%%%%%%%%%%%%%%%%%%%%%%%%%%%%%%%%%%%%%%%
% Lecture date: 19-11-05
%%%%%%%%%%%%%%%%%%%%%%%%%%%%%%%%%%%%%%%%%%%%%%%%%%%%%%%%%%%%%%%%%
\chapter{Standard Model}
We will be looking at the gauge group $\SU(2)_\text{L} \times \Uni(1)_\text{Y}$. This will be spontaneously broken into $\Uni(1)_{\text{EM}}$. Strong interaction will add $\SU(3)_\text{C}$ and it doesn't get spontaneously broken.

\section{Leptons}
Reintroduce fermions
\begin{align*}
   \psi &= \begin{pmatrix} \chi_\text{L} \\ \chi_\text{R} \end{pmatrix} \\
   \psi_\text{L} &= \begin{pmatrix} \chi_\text{L} \\ 0 \end{pmatrix}
\end{align*}
$\SU(2)_\text{L}$ only interacts with $\psi_\text{L}$.

In Chiral representation
\begin{align}
   \gamma^5 &= \begin{pmatrix} - \id_2 & 0 \\ 0 & \id_2 \end{pmatrix}
   \shortintertext{The projection operators}
   P_{L} &= \frac{1}{2} (\id_2 - \gamma^5) = \begin{pmatrix} \id_2 & 0 \\ 0 & 0\end{pmatrix} \\
   P_{R} &= \frac{1}{2} (\id_2 + \gamma^5) = \begin{pmatrix} 0 & 0 \\ 0 & \id_2 \end{pmatrix}
\end{align}
with $P_L^2 = P_L$ and $P_L + P_R = \id$.

$\chi_{\text{L}, \text{R}}$ two-component Weyl spinors.
\begin{align*}
   \left( \psi_\text{L} \right)^\dagger = (P_\text{L} \psi)^\dagger = \psi^\dagger P_\text{L}^\dagger = \psi^\dagger P_\text{L}\\
   \overline{\psi}_\text{L} = (\psi_L)^\dagger \gamma^0 = \psi^\dagger P_L \gamma^0 = \overline{\psi} P_R
\end{align*}

Introduce a doublet of left-handed particles
\begin{align*}
   L &= \begin{pmatrix} \nu_\text{L} \\ e^-_\text{L} \end{pmatrix} \\
   \nu_\text{L} &= (\psi_\nu)_\text{L} =  \begin{pmatrix} \chi_{\nu \text{L}} \\ 0 \end{pmatrix}    \\
   e_\text{L}^- &= (\psi_{e^-})_\text{L} = \begin{pmatrix} \chi_{e^- \text{L}} \\ 0 \end{pmatrix} \\
   e_\text{L} &= (\psi_e)_\text{L} = \frac{1}{2} \left( \id - \gamma_5 \right) \psi_e \\
   \shortintertext{There is no right-handed neutrino in this theory (singlet).}
   R &= e_\text{R} = P_\text{R} (\psi_e) = (\psi_e)_\text{R}
\end{align*}

Under $\Uni(1)_\text{Y}$ the hypercharges are define as
\begin{align}
   {Y}(L) &= -1 \\
   {Y}(R) &= -2
\end{align}
Here each component of $L$ has $Y=-1$. They are chosen so that 
\begin{align*}
   Q_{\text{EM}} = T^3_\text{L} + \frac{1}{2} Y
\end{align*}
with $T^3_\text{L} = \frac{1}{2} \tau^3$ a generator of $\SU(2)$.

\begin{align*}
   T^3_L = \begin{pmatrix} \frac{1}{2} & 0 \\ 0 & -\frac{1}{2} \end{pmatrix} \\
   T^3_L L  = \begin{pmatrix} \frac{1}{2} \nu_L \\ -\frac{1}{2} e^-_L\end{pmatrix}
\end{align*}

Take a look at the examples
\begin{align*}
   Q_{\text{EM}}(\nu_\text{L}) &= T^3_\text{L}(\nu_\text{L}) + \frac{1}{2} Y(\nu_\text{L}) = \frac{1}{2} - \frac{1}{2} = 0 \\
   Q_{\text{EM}}(e^-_\text{L}) &= T^3_\text{L}(e_\text{L}) + \frac{1}{2} Y(e_\text{L}) = -\frac{1}{2} - \frac{1}{2} = -1 \\
   Q_{\text{EM}}(e^-_\text{R}) &= T^3_\text{L}(e_\text{R}) + \frac{1}{2} Y(e_\text{R}) = 0 - 1 = -1
\end{align*}
$T_L$ only acts on left handed particles under $\SU(2)$
\begin{align*}
   \psi_{e_R} &= e^{i(0)} \psi_{e_R}  \\
   \comm{T^a}{Y} &= 0
\end{align*}

Normal kinetic term in Dirac theory
\begin{align*}
   \lag_\text{kin} &= \bar\psi i \gamma^\mu D_\mu (P_R + P_L) \psi \\
   \shortintertext{To split this using the anti-commutator of $\gamma^5$ and other gamma matrices, $\gamma^\mu P_L = P_R \gamma^\mu$}
                   &= \bar\psi_L i \gamma^\mu D_\mu \psi_L + \bar\psi_R i \gamma^\mu D_\mu \psi_R
\end{align*}

Thus kinetic terms for leptons are
\begin{align}
   \lag_{\text{leptons}}^{\text{kin}} &= \bar{R} i \gamma^\mu D'_\mu R + \bar{L} i \gamma^\mu D_\mu L  \\
   % \shortintertext{$D'_\mu$ only acts on $R$ and $Y(R) = -2$}
   \begin{split}
    D'_\mu &= \partial_\mu - \frac{ig'}{2} Y B_\mu  \\
          &= \partial_\mu + ig' B_\mu 
   \end{split} \\
   \begin{split}
    D_\mu &= \partial_\mu - \frac{ig'}{2}Y B_\mu - ig \frac{\tau^a}{2} W^a_\mu \\
         &= \partial_\mu + \frac{i}{2} g' B_\mu - ig \frac{\tau^a}{2} W_\mu^a
   \end{split}
\end{align}

%%%%%%%%%%%%%%%%%%%%%%%%%%%%%%%%%%%%%%%%%%%%%%%%%%%%%%%%%%%%%%%%%
% Lecture date: 19-11-11
%%%%%%%%%%%%%%%%%%%%%%%%%%%%%%%%%%%%%%%%%%%%%%%%%%%%%%%%%%%%%%%%%

In electromagnetic $\Uni(1)_\text{EM}$ the charge generator $Q$ with $Q(e^-) = -1$. Covariant derivative
\begin{align}
   D_\mu = \partial_\mu - ie Q A_\mu
\end{align}

In $\Uni(1)_\text{Y}$ the gauge boson $B_\mu$. Field strength tensor $F^\text{Y}_{\mu\nu} = \partial_\mu B_\mu - \partial_\nu B_\mu$. Charge generator $Y/2$. Covariant derivative 
\begin{align}
   D_\mu = \partial_\mu - \frac{ig'}{2} Y B_\mu
\end{align}

In Dirac Lagrangian there is a mass term $m\bar\psi \psi$.
\begin{align*}
   \lag_m &= m\bar\psi \psi = m \bar\psi (P_R + P_L) \psi \\
          &= m (\bar\psi P_R \psi + \bar\psi P_L \psi) \\
          &= m (\bar\psi P_R^2 \psi + \bar\psi P_L^2 \psi) \\
          &= m (\bar\psi_L \psi_R + \bar \psi_R \psi_L)
\end{align*}
$\bar\psi \psi$ mass term is called Dirac mass term. There is only left handed neutrino, so we cannot write Dirac mass term for neutrino.  Although $\psi_L$ and $\psi_R$ transform differently under Lorentz transformation, $\bar\psi \psi$ is still Lorentz invariant.

Recall that $\Uni(1)_\EM$ in QED $\psi \mapsto \psi'=\euler^{i\alpha(x)Q}\psi$. If $Q$ is same for left and right-handed components, $\bar\psi \psi$ is gauge invariant. If left and right have different hypercharges $Y$, then $\bar\psi_R \psi_L$ not $\Uni(1)_Y$ gauge invariant.

How about $\SU(2)_\text{L}$ gauge invariance of $\bar\psi_L \psi_R$? Under $\SU(2)$
\begin{align*}
   R &\mapsto R' = R \\
   L &\mapsto L' = \euler^{i\alpha_a(x) \tau^a /2} L
\end{align*}
So it's obvious that $\bar\psi_R \psi_L$ not $\SU(2)_\text{L}$ gauge invariant.

Any fermionic mass term vanishes, if we require $\SU(2)_\text{L} \times \Uni(1)_\text{Y}$ gauge invariance. Spontaneous symmetry breaking to let fermion gain mass.

\section{Add scalars}
Introduce a Higgs doublet
\begin{align}
   \Phi = \begin{pmatrix} \phi^+ \\ \phi^0 \end{pmatrix}
\end{align}
The superscripts denote the charge the field carries.

From previous section $Q_\EM = T^3_L + \frac{1}{2}Y $, so 
\begin{align*}
   Q_\EM(\phi^+) &= +1 =  \frac{1}{2} + \frac{1}{2} Y \Leftrightarrow Y=+1 \\
   Q_\EM(\phi^0) &= 0 = -\frac{1}{2} + \frac{1}{2} Y \Leftrightarrow Y=+1
\end{align*}
Together $Y(\Phi) = + 1$

Lagrangian
\begin{align}
   \begin{split}
    \lag_\text{scalar} &= (D_\mu \Phi)^\dagger (D^\mu \Phi) - V(\Phi^\dagger \Phi) \\
   D_\mu &= \partial_\mu - \frac{ig'}{2} B_\mu - \frac{ig}{2} \tau_i W^i_\mu \\
   V &= \mu^2 \Phi^\dagger \Phi + \lambda (\Phi^\dagger \Phi)^2 \\
   \Phi^\dagger &= \begin{pmatrix} \phi^+ \\ \phi^0 \end{pmatrix}^\dagger = \begin{pmatrix} \phi^+ & \phi^0\end{pmatrix}^* = \begin{pmatrix} \phi^- & (\phi^0)^* \end{pmatrix}
   \end{split}
\end{align}

\section{Coupling of Scalars and Fermions}
Recall that simple complex scalar $\phi^0 \in \Co$, $\SU(2)$ singlet. Under Lorentz transformations (per definition) $\phi^0(x) \mapsto \phi^0 (x)$. $\bar\psi \psi$ is also Lorentz invariant. It means $\phi^0\bar\psi \psi$ is Lorentz invariant. This type of interaction is called Yukawa interaction.
\begin{align*}
   \lag_\text{Yukawa} &= - y_e \left( \overline{R} \Phi^\dagger L + \overline{L} \Phi R \right) \\
   &\feynmandiagram[small, inline=(a.base), horizontal=a to x]{ a --[scalar] x --[fermion] f1, x -- [anti fermion] f2,};
\end{align*}

How to combine $\Phi$, $L$ and $R$? 
\begin{align*}
   Y(\Phi) &= +1  \\
   Y(L) &= -1 \\
   Y(R) &= -2 \\
   Y(\bar{L}) &= +1 
\end{align*}

Gauge transformation of interaction term 
\begin{align*}
   \phi \bar\psi \psi \mapsto \phi' \bar\psi' \psi' = \euler^{i\alpha(Q(\phi) + Q(\bar\psi) + Q(\psi))} \phi \bar\psi \psi
\end{align*}
Gauge invariance means the sum of hypercharges is zero. $\Phi \bar{L} R$ is $\Uni(1)_Y$ gauge invariant. 

What if we want $\SU(2)_\text{L} \times \Uni(1)_\text{Y}$ gauge invariant. 

$R$ is $\SU(2)$ invariant. $\Phi$ is $\underline{2}$ under $\SU(2)$. $L$ and $\bar L$ are $\underline{2}$ under $\SU(2)$. For $\SU(2)$ these two kinds of representations are the same $\bar{\underline{2}} = \underline{2}$.
\begin{align*}
\Phi \bar L: \underline{2} \otimes \bar{\underline{2}} = \underline{3} \oplus \underline{1}_A
\end{align*}

$\Phi = \begin{pmatrix} \phi^+ \\ \phi^0 \end{pmatrix} = \begin{pmatrix} \phi_1 \\ \phi_2 \end{pmatrix}$.
$\underline{1}$ is $\SU(2)$ singlet. Need antisymmetric combination of $\Phi \bar L$ to get $\SU(2)$ singlet.

%%%%%%%%%%%%%%%%%%%%%%%%%%%%%%%%%%%%%%%%%%%%%%%%%%%%%%%%%%%%%%%%%
% Lecture date: 19-11-12
%%%%%%%%%%%%%%%%%%%%%%%%%%%%%%%%%%%%%%%%%%%%%%%%%%%%%%%%%%%%%%%%%

In components
\begin{align*}
   \overline{L} \Phi R &= \begin{pmatrix} \overline{\nu}_L & \overline{e}_L^- \end{pmatrix} \begin{pmatrix} \phi^+ \\ \phi^0 \end{pmatrix} e_R \\
   &= \overline{\nu}_L \phi^+ e_R + \overline{e}_L^- \phi^0 e_R
   \shortintertext{here $\sum Y_i=0$}
   \overline{R} \Phi^+ L &= \overline{e}^-_R \begin{pmatrix} \phi^- & \left(\phi^0 \right)^*\end{pmatrix} \begin{pmatrix} \nu_L \\ e_L^-\end{pmatrix} \\
   &= \overline{e}_R^- \phi^- \nu_L + \overline{e}_R^- \left( \phi^0 \right)^* e_L^-
\end{align*}

The \sm does not contain $\nu_R$. But let's consider $\nu_R$ anyway
\begin{align*}
   Q_\EM &= T^3_L + \frac{1}{2} Y \\
   0 &= 0 + \frac{1}{2} Y(\nu_R)
\end{align*}
So it doesn't couple to $B_\mu$.

How about Dirac mass term?
\begin{align*}
   \overline{\nu}_L \nu_R + \overline{\nu}_R \nu_L
\end{align*}
It is not $\Uni(1)_\text{Y}$ or $\SU(2)_\text{L}$ invariant. Not a huge problem since it was the same for electron before we introduce Higgs mechanism.

Cannot write interaction like $\overline{L} \nu_R \Phi$ since the hypercharge is not zero. Instead
\begin{align*}
   \tilde{\Phi} &= i \tau_2 \Phi^* = \begin{pmatrix} \phi^0 \\ -\phi^- \end{pmatrix} \\
   Y(\tilde{\Phi}) &= -1 \\
   \lag_{\nu_L} &= \overline{L} \tilde{\Phi} \nu_R  + \overline{\nu}_R \tilde{\Phi}^\dagger L
   \shortintertext{Dirac mass term for neutrino. If $\nu_R$ exists, this is a possible mass term.}
   \overline{L} \tilde{\Phi} \nu_R &= \begin{pmatrix} \overline{\nu}_L & \overline{e}^-_L \end{pmatrix} \begin{pmatrix} (\phi^0)^* \\ -\phi^- \end{pmatrix} \nu_R  \\
   &= \overline{\nu} (\phi^0)^* \nu_R - \overline{e}_L^- \phi^- \nu_R
\end{align*} 

\section[Spontaneous Symmetry Breaking (Mass and Mixing of Gauge Bosons)]{Spontaneous Symmetry Breaking (Mass and Mixing of Gauge Bosons)\footnote{see also in Cheng and Li, Ch.11}}
In the early universe $\mu = \mu(T)$ and the symmetry got broken. Consider the case $\mu^2 < 0$.
\begin{align}
   \expval{\Phi} = \frac{1}{\sqrt{2}} \begin{pmatrix} 0 \\ v \end{pmatrix}
\end{align}
with $v \in \R$ and $[v] = 1$. Field $\phi^0$ has $Q_\EM = 0$. $Y(\phi^0) \neq 0$ breaks the $\Uni(1)_Y$ and $T_L^3(\phi^0) \neq 0$ breaks $\SU(2)_L$.

As before the field shift is 
\begin{align*}
   \Phi &= U^{-1}(\pmb\xi) \frac{1}{\sqrt{2}} \begin{pmatrix} 0 \\ v+\eta(x) \end{pmatrix} \\
   U(\pmb\xi) &= \exp(-i \pmb\xi \cdot \pmb\tau / (2v))
\end{align*}
$\eta(x), \pmb{\xi}(x) \in \R$. The transformation $U^{-1}(\pmb \xi)$ has three parameters and thus give $\Phi$ four degrees of freedom. It has the same form as $\SU(2)$ gauge transformation. Because of gauge symmetry, this transformation gets cancelled. 
\begin{align}
   \Phi &\mapsto \Phi' = U(\xi) \Phi = \frac{1}{\sqrt{2}} \begin{pmatrix} 0 \\ v + \eta(x) \end{pmatrix} \label{math:PhiTrafo}\\ 
   L &\mapsto L' = U(\pmb\xi) L \\
   W_\mu &\mapsto W'_\mu \\
   \pmb\tau \cdot \pmb W'_\mu &= U(\pmb\xi) \left[ \pmb \tau \cdot \pmb{W}_\mu - \frac{i}{g} U^{-1}\partial_\mu U \right]
\end{align}

Insert transformed field (\ref{math:PhiTrafo}) into $\lag$ and see the physical interpretation. Yukawa coupling
\begin{align*}
   - y_e \overline{L} \Phi R &= - y_e \begin{pmatrix} \overline{\nu}_e & \overline{e}_L^- \end{pmatrix} \frac{1}{\sqrt{2}} \begin{pmatrix} 0 \\ v + \eta(x)\end{pmatrix} e_R \\
   &= - \frac{y_e}{\sqrt{2}} \overline{e}_L^- (v+\eta(x)) e_R
\end{align*}
Thus $m_e = y_e v / \sqrt{2}$. Knowing $m_e$ we can fix $y_e v$. It does not predict the mass of electron, but we can accommodate. We do predict $m_\nu = 0$ though.

There is an extra term
\begin{align*}
   - \frac{y_e}{\sqrt{2}} \overline{e}_L^- e_R \eta(x)
\end{align*}
It indicates the coupling of Higgs is direct proportional to mass of fermion. We found $y_e \sim 10^{-6}$, so it is then not highly unlikely to observe Higgs decay into electrons at LHC. We have already seen Higgs decay into tau and bottom quarks.

\begin{align}
   \lag_\text{scalar} = \frac{1}{2} (\partial_\mu \eta) (\partial^\mu \eta) + V \left(\left(\frac{v+\eta}{\sqrt{2}}\right)^2 \right)
\end{align} 

Write 
\begin{align*}
   \Phi &= \frac{v + \eta}{\sqrt{2}} \chi \\ 
   \chi &= \begin{pmatrix} 0 \\ 1\end{pmatrix}
\end{align*}
Multiply $(D_\mu \Phi)^\dagger (D^\mu \Phi)$ out but focus only on terms without derivative
\begin{align*}
   D_\mu &= \partial_\mu - \frac{i}{2} g' B_\mu - ig \frac{\tau^i}{2} W_\mu^i \\
   \lag  &\subset \frac{(v+\eta)^2}{8} \chi^\dagger \left[ (g'B_\mu^\dagger + g \tau^i W^i _\mu) \cdot (g' B^\mu + g \tau^i W^{i \mu})\right] \chi
\end{align*}
Terms with one $\tau^1$ and $\tau^2$ vanish
\begin{align*}
   \chi^\dagger \tau^1 \chi &= 0\\
   \chi^\dagger \tau^2 \chi &= \begin{pmatrix} 0 & 1\end{pmatrix} \begin{pmatrix} 0 & -i \\ i & 0 \end{pmatrix} \begin{pmatrix} 0 \\ 1 \end{pmatrix} = 0 \\
   \chi^\dagger \tau^3 \chi &= \begin{pmatrix} 0 & 1\end{pmatrix} \begin{pmatrix} 1 & 0 \\ 0 & -1 \end{pmatrix} \begin{pmatrix} 0 \\ 1\end{pmatrix} = -1 \\
\end{align*}
Focus on $\sim v^2$ terms, they are bilinear in gauge boson fields and thus mass terms.
\begin{align}
   \lag \subset \frac{1}{8} v^2 \left[ (g'B_\mu - g W^3_\mu) (g' B^\mu-gW^{3 \mu}) + g^2 (W_\mu^1)^2 + g^2 (W_\mu^2)^2 \right]
\end{align}
with $M^2_{W^1} = M^2_{W^2} = {g^2 v^2}/ {8}$.

% TODO: WHAT?????
Using the familiar formula $Q = T_3 + \frac{1}{2} Y$ for the $\SU(2)$ gauge boson $W^{1,2,3}$. They form a $\SU(2)$ triplet $T_3 = \pm1, 0$. $W^3$ has zero electric charge. $B_\mu$ as $\SU(2)$ singlet has zero electric charge and hypercharge. 

Focus on the electric neutral part
\begin{align*}
   \lag &\subset \frac{v^2}{8} \left[  g'^2 B_\mu B^\mu + g^2 W_\mu^3 W^{\mu 3} + g'g B_\mu W^{\mu 3} + g'gW^3_\mu B^\mu \right]
   \shortintertext{It has $W_\mu B^\mu$ mixing terms. Rewritten in matrix form}
        &= \frac{v^2}{8} \begin{pmatrix} B_\mu & W_\mu^3 \end{pmatrix} \begin{pmatrix} g'^2 & gg' \\ gg' & g^2 \end{pmatrix} \begin{pmatrix} B^\mu \\ W^{3\mu}\end{pmatrix} \\
        &= \frac{1}{2} \begin{pmatrix} Z_\mu & A_\mu \end{pmatrix} \begin{pmatrix} M_Z^2 & 0 \\ 0 & 0\end{pmatrix} \begin{pmatrix} Z^\mu \\ A^\mu \end{pmatrix}
\end{align*}
Using the properties of $2\times 2$ matrices
\begin{align}
   \det &= \lambda_1 \cdot \lambda_2 \\
   \tr &= \lambda_1 + \lambda_2
\end{align}
Obviously here $\det = 0$ and $\tr = g^2 + g'^2$. 

Eigenvectors are
\begin{align}
   \begin{split}
      Z_\mu &= \frac{-g W_\mu^3 + g' B_\mu}{\sqrt{g^2 + g'^2}} \\
      A_\mu &= \frac{g B_\mu + g' W_\mu^3}{\sqrt{g^2 + g'^2}}
   \end{split}\label{math:AZ}\\
   \begin{split}
    M_A &= 0  \\
   M_Z &= \frac{v}{2} \sqrt{g^2 + g'^2 }
   \end{split}
\end{align}
Remember real fields have mass term in the form $\frac{1}{2} m^2 \phi^2$.

Go back to $W^1_\mu$ and $W^2_\mu$. Define
\begin{align}
   W^\pm &= \frac{1}{\sqrt{2}} (W_\mu^1 \pm W_\mu^2) \\
   W^+_\mu W^{- \mu} &= \frac{1}{2} \left[ (W_\mu^1)^2 + (W_\mu^2)^2 \right]
\end{align}

$\pmb W$ is a triplet $(W^+, W^0,  W^-)^T $. The superscripts denote the hypercharge ($T^3$). $W^3_\mu$ has zero hypercharge and thus zero electric charge.

% Use $Q = T_3 + \frac{1}{2} Y$.
Mass term $\frac{1}{2} m^2 (W_\mu^+ W^{- \mu} + W^-_\mu W^{\mu +})$
\begin{align*}
   \frac{1}{2} M_W^2 &= \frac{1}{8} g^2 v^2 \\
   M_W &= \frac{1}{2} g v
\end{align*}

Go back to leptons
\begin{align*}
   \lag_{\text{kin}}^\text{leptons} &= \overline{R} i \gamma^\mu D'_\mu R + \overline{L} i \gamma^\mu D_\mu L \\
   D'_\mu &= \partial_\mu + i \frac{g'}{2} B_\mu \\
   D_\mu &= \partial_\mu + \frac{i}{2} g' B_\mu - ig \frac{\tau^i}{2} W_\mu^i
\end{align*}
Use equation~(\ref{math:AZ}) to rewrite $B_\mu = f(A_\mu, Z_\mu)$ and $W_\mu^3 = f'(A_\mu, Z_\mu)$.

Insert that into $D'_\mu$ and $D_\mu$ and then into $\lag_\text{kin}^\text{leptons}$. Also replace ${W'_\mu}^2 \mapsto W_\mu^{\pm}$
\begin{align*}
   \overline{L} i \gamma^\mu D_\mu L &= \overline{L} i \gamma^\mu  \left( \partial_\mu + \frac{i}{2} g' B_\mu - ig \frac{\tau^i}{2} W_\mu^i \right) L \\
   \tau_1 &= \begin{pmatrix} 0 & 1 \\ 1 & 0 \end{pmatrix}, \quad \tau_2 = \begin{pmatrix} 0 & -i \\ i & 0\end{pmatrix} \\
   \tau_1 W^1_\mu + \tau^2 W_\mu^2 &= \begin{pmatrix} 0 & W_\mu^1 - iW_\mu^2 \\ W_\mu + iW_\mu^2 & 0 \end{pmatrix}
\end{align*}

Charged current
\begin{align*}
   A &= \overline{L} \gamma^\mu \left( \tau^1 W^1_\mu + \tau^2 W_\mu^2 \right) L \\
     &= \frac{g}{2} \begin{pmatrix} \overline{\nu}_L & \overline{e}_L \end{pmatrix} \gamma^\mu \begin{pmatrix} 0 & \sqrt{2} W_\mu^+ \\ \sqrt{2} W_\mu^- & 0 \end{pmatrix} \begin{pmatrix} \nu_L \\ e_L \end{pmatrix} \\
   &= \frac{g}{2} \begin{pmatrix} \overline{\nu}_L & \overline{e}_L \end{pmatrix} \gamma^\mu \begin{pmatrix} \sqrt{2} W_\mu^+ e_L \\ \sqrt{2} W_\mu^- \nu_L\end{pmatrix} \\
   &= \frac{g}{\sqrt{2}} \left[ \overline{\nu}_L W_\mu^+ \gamma^\mu e_L + \overline{e}_L W^-_\mu \gamma^\mu \nu_L \right]
\end{align*}
Diagrammatically
\begin{align*}
   \feynmandiagram[layered layout, inline=(v.base), horizontal=v to f]{
      i1[particle=\(\nu\)] --[anti fermion] v, 
      i2[particle=\(e\)] --[fermion] v --[photon] f[particle=\(W\)],
   }; \sim ig \gamma^\mu P_L
\end{align*}

Neutrino-electron scattering has the cross section $\sim g^4 / M^4_W$
\begin{align*}
   \feynmandiagram[horizontal=v1 to v2]{
      i1[particle=\(\nu_e\)] --[fermion] v1 --[fermion] f1[particle=\(e^-\)],
      i2[particle=\(e^-\)] --[fermion] v2 --[fermion] f2[particle=\(\nu_e\)],
      v1 --[photon, edge label=\(W\)] v2
   };
\end{align*}

Fermi constant
\begin{align}
   G_F = \frac{\sqrt{2} g^2}{8 M^2_W}
\end{align}

$\lag_\text{kin}^\text{leptons}$ can be extended to muons. Then we can draw diagram for muon decay. Decay rate $\Gamma \sim G_F^2$. $\tau_\mu \sim \SI{2}{\micro\s}$. $G_F \sim 10^{-5}/m^2_\mu$
\begin{align*}
   \feynmandiagram[horizontal=i1 to v1, medium, layered layout]{
      i1[particle=\(\mu\)] -- [fermion]v1,
      v1 -- [fermion] f3[particle=\(\nu_\mu\)],
      v1 -- [photon, edge label=\(W\)] v2,
      {[same layer] v2, f3},
      v2 --[fermion] f1[particle=\(e^-\)],
      v2 --[anti fermion] f2[particle=\(\bar\nu_e\)],
      {[same layer] f1, f2},
   };
\end{align*}
%%%%%%%%%%%%%%%%%%%%%%%%%%%%%%%%%%%%%%%%%%%%%%%%%%%%%%%%%%%%%%%%%
% Lecture date: 19-11-18
%%%%%%%%%%%%%%%%%%%%%%%%%%%%%%%%%%%%%%%%%%%%%%%%%%%%%%%%%%%%%%%%%
We have three parameters in gauge sector $(g, g', v) \leftrightarrow (\alpha_{\EM}, G_\text{F}, G_{\text{NC}})$. We measured $\alpha_\EM = 1/137$, $G_\text{F}$ in $\mu$-decay and $G_\text{NC}$ in neutral current interactions.

Equations (\ref{math:AZ}) are just basis transformation and can be parametrized with one parameter $\theta_W$.
\begin{align}
   \tan(\theta_W) &= \frac{g'}{g}  \\
   \sin(\theta_W) &= \frac{g'}{\sqrt{g^2 + g'^2}} \\
   \cos(\theta_W) &= \frac{g}{\sqrt{g^2 + g'^2}}
\end{align}

Equations (\ref{math:AZ}) become
\begin{align}
   A_\mu &= \cos(\theta_W) B_\mu + \sin(\theta_W) W_\mu^3 \\
   Z_\mu &= \sin(\theta_W)B_\mu - \cos(\theta_W) W_\mu^3
\end{align}
The transformation can be inverted 
\begin{align}
   B_\mu &= \cos(\theta_W) A_\mu + \sin(\theta_W) Z_\mu \\
   W_\mu^3 &= \sin(\theta_W)A_\mu - \cos(\theta_W) Z_\mu
\end{align}
The angle (parameter) $\theta_W$ is called Weinberg angle or \underline{electroweak mixing angle}.

Rewrite the $\lag^\text{lepton}$ in particular neutral current and factor out $A_\mu, Z_\mu$
\begin{align*}
   \lag^\text{lepton} &= A_\mu \frac{gg'}{\sqrt{g^2 + g'^2}} \left[ \bar{e}_R \gamma^\mu e_R + \bar{e}_L \gamma^\mu e_L \right]  \\ 
   &\quad + Z_\mu \frac{1}{2\sqrt{g^2 + g'^2}} \left[ g'^2 \left( 2 \bar{e}_R \gamma^\mu e_R + \bar{e}_L \gamma^\mu e_L + \bar{\nu}_L \gamma^\mu \nu_L \right) -g^2 \left( \bar{e}_L \gamma^\mu e_L - \bar{\nu}_L \gamma^\mu \nu_L \right) \right]
\end{align*}
First two terms are coupling of electron and photon and the coupling constant is just electric charge.
\begin{align}
   e = \frac{gg'}{\sqrt{g^2 + g'^2}} \approx 0.3
\end{align}

\begin{align*}
   \feynmandiagram[vertical=v1 to v2, baseline=(v2.base)]{
      i1[particle=\(\nu_\mu\)] --[fermion] v1 --[fermion] f1[particle=\(\mu\)],
      i2[particle=\(e\)] --[fermion] v2 --[fermion] f2[particle=\(\nu_e\)],
      v1 --[photon, edge label=\(W\)] v2
   }; & \quad \text{charged current} \\
   \feynmandiagram[vertical=v1 to v2, baseline=(v2.base)]{
      i1[particle=\(\nu_\mu\)] --[fermion] v1 --[fermion] f1[particle=\(\nu_\mu\)],
      i2[particle=\(e^-\)] --[fermion] v2 --[fermion] f2[particle=\(e^-\)],
      v1 --[photon, edge label=\(Z^0\)] v2
}; & \quad \text{neutral current}
\end{align*}
Charged current interaction involves $W^{\pm}$. Neutral current interaction involves $A_\mu$ and $Z_\mu^0$.

One can experimentally determine $g, g', v$ (actually $G_F$, $e^2$ and $\sin^2(\theta_W)$)
\begin{align*}
   e &= g \sin(\theta_W) = g' \cos(\theta_W)
     &&\Rightarrow \sin^2(\theta_W) \approx 0.22 \\
   M_W^2 &= \frac{1}{4} g^2 v^2 = \frac{e^2}{\sin^2(\theta_W)} \frac{1}{4\sqrt{2} G_F}
         &&\Rightarrow M_W \approx \SI{80}{\giga \eV}  \quad
   M_Z \approx \SI{90}{\giga \eV}
\end{align*}
More precisely one should also consider electroweak quantum corrections (loop corrections).

\begin{align*}
   & A_\mu \left[ \bar{e}_R \gamma^\mu e_R + \bar{e}_L \gamma^\mu e_L \right] \frac{gg'}{\sqrt{g^2 + g'^2}} \\
   =& A_\mu \left[ \bar{e}_R \gamma^\mu e_R + \bar{e}_L \gamma^\mu e_L \right] e_\text{el}\\
   =& A_\mu \left[ \bar{e} \gamma_\mu P_R e + \bar{e } \gamma^\mu P_L e \right] e_\text{el} \\
   =& A_\mu (\bar{e} \gamma^\mu e)
\end{align*}
It is Lorentz four vector.

For $Z_\mu$ it is different. Left-handed electrons are coupled differently from right-handed ones.
\begin{align*}
   Z_\mu \bar{e}_R \gamma^\mu e_R \sim \frac{2 g'^2}{2\sqrt{g^2 + g'^2}} \\
   Z_\mu \bar{e}_L \gamma^\mu e_L \sim \frac{g'^2 - g^2}{2 \sqrt{g^2 + g'^2}}
\end{align*}
Write this in one equation
\begin{align}
   &\frac{1}{2} Z_\mu \left[  \bar{e} \gamma^\mu \left( P_R \frac{2g'^2}{\sqrt{g'^2 + g^2}} + P_L \frac{g'^2 - g^2}{\sqrt{g^2 + g'^2}} \right)\right]e_\text{el}
   \shortintertext{suppress the constants into $C_A$ and $C_V$}
   =& \bar{e} \gamma^\mu C_V e + \bar{e} \gamma^\mu \gamma^5 C_A e
\end{align}

The ratio of these two coupling is
\begin{align}
   \frac{C_V}{C_A} = \frac{3g'^2 - g^2}{g^2 + g'^2} = -1 + 4 \sin^2(\theta_W)
\end{align}

%%%%%%%%%%%%%%%%%%%%%%%%%%%%%%%%%%%%%%%%%%%%%%%%%%%%%%%%%%%%%%%%%
% Lecture date: 19-11-18
%%%%%%%%%%%%%%%%%%%%%%%%%%%%%%%%%%%%%%%%%%%%%%%%%%%%%%%%%%%%%%%%%
\section{Quarks}
Introduce a quark doublet (under $\SU(2)$)
\begin{align}
   Q = \begin{pmatrix} u_L \\ d_L \end{pmatrix}
\end{align}
$u_L$ and $d_L$ are actually triplets under $\SU(3)_\text{C}$

\begin{align*}
   Q_\text{el}(u_L) &= + \frac{2}{3}  \leftrightarrow  Y(u_L)  = \frac{1}{3} \\
   Q_\text{el}(d_L) &= - \frac{1}{3}  \leftrightarrow Y(d_L)  = \frac{1}{3}
\end{align*}
Together
\begin{align}
   Y(Q) = + \frac{1}{3}
\end{align}

$u_R$ and $d_R$ are $\SU(2)$ singlets.
\begin{align*}
   Q_\text{el}(u_R) &= + \frac{2}{3}  \leftrightarrow  Y(u_L) = \frac{4}{3} \\
   Q_\text{el}(d_R) &= - \frac{1}{3}  \leftrightarrow Y(d_L) = \frac{4}{3}
\end{align*}

\begin{align*}
 \begin{tabular}{cccc}
   \toprule
& $\SU(3)_\text{C}$ & $\SU(2)_\text{L}$ & $\Uni(1)_\text{Y}$ \\
\midrule
   $e_L$ &  $1$ & $2$ & $-1$ \\
$u_R$ & $3$ & $1$ & $4/3$ \\
$d_R$ & $3$ & $1$ & $-2/3$ \\
$Q$ & $3$ & $2$ & $1/3$ \\
$\Phi$ & $1$ & $2$ & $1$ \\
\bottomrule
\end{tabular}
\end{align*}

Kinetic terms for quarks in Lagrangian 
\begin{align}
   \lag_\text{kin}^\text{quarks} &= \bar{Q} i \gamma_\mu D^\mu Q + \bar{u}_R i \gamma^\mu D^\mu u_R + \bar{d}_R i \gamma_\mu D^\mu d_R \\
   D^\mu &= \partial_\mu - \frac{ig'}{2}YB_\mu - ig \frac{\tau^i W_\mu^i}{2} - ig_3 T^a \gf^{\mu a}
\end{align}

Higgs Lagrangian unchanged. Replace $(B_\mu, W_\mu^3) \mapsto (A_\mu, Z_\mu)$

\paragraph{Neutral current}
\begin{align}
   \frac{1}{2} Z^\mu \frac{1}{\sqrt{g'^2 + g^2}} \left[ g'^2 \left( \frac{4}{3} \bar{u}_R \gamma_\mu u_R - \frac{2}{3}\bar{d}_R \gamma_\mu d_R + \frac{1}{3} \bar{u}_L \gamma_\mu u_L + \frac{1}{3} \bar{d}_L \gamma_\mu \bar{d}_L \right) + g^2 \left( \bar{u}_L \gamma_\mu u_L - \bar{d}_L \gamma_\mu d_L \right) \right]
\end{align}
$A_\mu$ is vector like again with charges $2/3$ ($u$) and $-1/3$ ($d$).

Neutral currents are always diagonal in quark fields. 

Yukawa coupling of quarks to Higgs
\begin{align*}
   y_d \bar{Q} \Phi d_R
\end{align*}
Do the hypercharge add up to zero?
\begin{align*}
   Y(\bar{Q}) + Y(\Phi) + Y(d_R) = -\frac{1}{3} + 1 - \frac{2}{3} = 0
\end{align*}

Analogous terms with $u_R$ does \underline{not} work use $\nu_R$ trick $\tilde{\Phi} = i\sigma_2 \Phi^*$
\begin{align*}
   y_u \bar{Q} \tilde{\Phi} u_R
\end{align*}

\paragraph{Charged Current}

\begin{align}
   A_\mu \left[ \bar{u}_R \gamma^\mu u_e + \bar{u}_L \gamma^\mu u_L \right] e e_u = A_\mu \bar{u} \gamma^\mu u
\end{align}

As before, same $W^\pm$ interaction only involved $P_L$
\begin{align*}
   & \bar{L} \gamma^\mu W_\mu L \\
   L &= P_L L \\
   \gamma^\mu P_L &= \frac{1}{2} \left( \gamma^\mu - \gamma^\mu \gamma^5 \right)
\end{align*}
$\gamma_\mu$ is vector interaction. $\gamma^\mu \gamma^5$ axial vector interaction. 
So charged current is V-A interaction. Parity is maximally violated.

Coupling $g$ is same for leptons and quarks.

Nuclear beta decay $n \rightarrow p e^- \bar{\nu}_e$
\begin{align*}
   \feynmandiagram[horizontal=i1 to v1, medium, layered layout]{
      i1[particle=\(d\)] -- [fermion]v1,
      v1 -- [fermion] f3[particle=\(u\)],
      v1 -- [photon, edge label=\(W\)] v2,
      {[same layer] v2, f3},
      v2 --[fermion] f1[particle=\(e^-\)],
      v2 --[anti fermion] f2[particle=\(\bar\nu_e\)],
      {[same layer] f1, f2},
   };
\end{align*}
This diagram and the one with muon have same coupling $G_F$. Only kinematics is different. Assuming $m_\nu = 0$ we can compare these two processes by measuring neutron lifetime and $\mu$-lifetime. The two Fermi constants are not the same!

%%%%%%%%%%%%%%%%%%%%%%%%%%%%%%%%%%%%%%%%%%%%%%%%%%%%%%%%%%%%%%%%%
% Lecture date: 19-11-19
%%%%%%%%%%%%%%%%%%%%%%%%%%%%%%%%%%%%%%%%%%%%%%%%%%%%%%%%%%%%%%%%%
It should be clear how to generalize current Lagrangian to other quarks
\begin{align*}
   Q_1 = \begin{pmatrix} u \\ d \end{pmatrix} \quad
   Q_2 &= \begin{pmatrix} c \\ s \end{pmatrix}  \quad 
   Q_3 = \begin{pmatrix} t \\ b \end{pmatrix}  \\
   \pmb{Q} &= (Q_1, Q_2, Q_3)  \\
   \pmb{u}_R &= (u_R, c_R, t_R) \\
   \pmb{d}_R &= (d_R, s_R, b_R)
\end{align*}

\begin{align*}
   \frac{4}{3} \bar{\pmb{u}}_R \gamma_\mu \pmb{u}_R Z^\mu 
   = \frac{4}{3} \left[ \bar{u}_R \gamma_\mu u_R + \bar{c}_R \gamma_\mu c_R + \bar{t}_R \gamma_\mu t_R \right] Z^\mu
\end{align*}

Neutral current interactions are unaffected by a unitary transformation in flavour space
\begin{align}
   \pmb{u}_R \mapsto \pmb{u}'_R = A \pmb{u}_R = \begin{pmatrix} u'_R \\ c'_R \\ t'_R \end{pmatrix}
\end{align}
Then the interaction term
\begin{align*}
   \bar{\pmb{u}}_R \gamma^\mu \pmb{u}_R &\mapsto \bar{\pmb{u}}'_R \gamma^\mu \pmb{u}'_R \\
                                        &= \bar{\pmb{u}}_R A^\dagger \gamma^\mu A \pmb{u}_R \\
                                        &= \bar{\pmb{u}}_R \gamma^\mu \pmb{u}_R
\end{align*}

The Yukawa interactions
\begin{align*}
   y_{ij}^d \bar{Q}_i \Phi d_{R_{j}} + y^u_{ij} \bar{Q}_j \tilde{\Phi} u_{R_j}
\end{align*}

Higgs interactions $\Phi \mapsto v$
\begin{align*}
   y^{d}_{ij} \cdot v = M^d_{ij}
\end{align*}
It is a $n \times n$ complex matrix. $n$ is the number of families. In \sm there are three families.

Charge current interactions
\begin{itemize}
   \item leptons
      \begin{align*}
         \lag_{\text{leptons}}^{\text{CC}} = \frac{g}{\sqrt{2}} \left[ \bar{\nu}_L W_\mu^\dagger \gamma^\mu e_L + \bar{e}_L W_\mu^- \gamma^\mu \nu_L \right]
      \end{align*}
   \item quarks
      \begin{align*}
         \lag_{\text{quarks}}^{\text{CC}} = \frac{g}{\sqrt{2}}\left[ \bar{u}_L W_\mu^\dagger \gamma^\mu d_L + \bar{d}_L W_\mu^- \gamma^\mu u_L \right]
      \end{align*}
      The matrices are not diagonal in flavour.
\end{itemize}

Focus on $M_{ij}$. It is not necessary symmetric. Mass terms are in form of $m(\bar{\psi}_L \psi_R + \bar{\psi}_R {\psi}_L)$

\paragraph{Theorem} $M_{ij}$ can be diagonalized by a bi-unitary transformation 
\begin{align}
   S^\dagger M T = M_d 
\end{align}
with $S$ and $T$ unitary. $M_d$ is diagonal and has positive eigenvalues.

\underline{Proof} Any matrix $M = H \cdot V$ with $H$ hermitian and $V$ unitary.
\begin{align*}
   (M M^\dagger )^\dagger = M M^\dagger
\end{align*}
In word, $MM^\dagger$ is hermitian. Here matrix can be diagonalized by unitary matrix $S$
\begin{align*}
   M^2_d &= S^\dagger (MM^\dagger) S  \\
   M_d^2 &= \diag(m_1^2, m_2^2, m_3^2)
\end{align*}

$S$ is unique up to a diagonal phase matrix.
\begin{align}
   F = \diag(\euler^{i\phi_1}, \euler^{i\phi_2}, \euler^{i\phi_3})
\end{align}

Replace $S \mapsto S' = SF$
\begin{align*}
   & (SF)^\dagger (MM^\dagger) (SF) \\
   =& F^\dagger S^\dagger (MM^\dagger) S F \\
   =& F^\dagger M_d^2 F \\
   =& M_d^2 
\end{align*}

\begin{align*}
   S^\dagger M T &= M_d \\
   (SF)^\dagger MT &= M_d \\
   S^\dagger MT &= F M_d 
\end{align*}
Use the phase freedom to have entries $m_i \geq 0$.

\underline{Define} $H = S M_d S^\dagger $
\begin{align*}
   H^\dagger = (SM_d S^\dagger)^\dagger = S M_d S^\dagger = H
\end{align*}
$H$ is hermitian.

\underline{Define} $V = H^{-1} M $ $V^\dagger = M^\dagger H^{-1}$

Compute
\begin{align*}
   V V^\dagger &= H^{-1} M M^\dagger H^{-1} \\
               &= H^{-1} S M_d^2 S^\dagger H^{-1} \\
               &= H^{-1} S M_d S^\dagger S M_d S^\dagger H^{-1} \\
               &= H^{-1} H H H^{-1} \\
               &= \id
\end{align*}
so $V$ is unitary.

\begin{align*}
   V &= H^{-1} M \\
   HV &= M \\
   H &= M V^\dagger
\end{align*}

\begin{align*}
   S^\dagger H S &= S^\dagger M V^\dagger S
   \shortintertext{LHS by definition is $M_d$}
   M_d &= S^\dagger M V^\dagger S \\
   M_d &= S^\dagger M T \quad \text{with } T = V^\dagger S
\end{align*}

Recall that our entries after spontaneous symmetry breaking involve 
\begin{align*}
  &M_{ij}^u \bar{u}_{L_i} U_{R_j} + M_{ij}^d \bar{d}_{L_i} d_{R_j}   \\
   =& \pmb{u}_L \underline{M}^u \pmb{u}_R + \pmb{d}_L \underline{M}^d \pmb{d}_R
\end{align*}

Call the transformation to mass eigenstates
\begin{align*}
  \pmb u_L = S_u \pmb u'_L \\
  \pmb d_L = S_d \pmb d'_L \\
  \pmb u_R = T_u \pmb u'_R \\
  \pmb d_R = T_d \pmb d'_R
\end{align*}
This transformation has no effect on neutral current interactions. (It does not live in spinor space). Charged current interactions on the other hand
\begin{align*}
   &\frac{g}{\sqrt{2}} \left[ \bar{u}_L W_\mu^\dagger \gamma^\mu d_L + \bar{d}_L W_\mu^- \gamma^\mu u_L \right] + \text{terms for $c$, $s$, $t$ and $b$} \\
   =& \frac{g}{\sqrt{2}} \left[ \bar{\pmb{u}}_L \gamma^\mu \pmb{d}_L W_\mu^\dagger + \bar{\pmb{d}}_L \gamma^\mu \pmb{u}_L W_\mu^- \right]
   \shortintertext{Charged current transforms to}
   =&\frac{g}{\sqrt{2}} \left[ \bar{\pmb{u}}'_L \gamma^\mu \left( S_u^\dagger S_d \right) \pmb{d}'_L W_\mu^\dagger + \bar{\pmb{d}}'_L \gamma^\mu \left( S_d^\dagger S_u \right) \pmb{u}'_L W_\mu^- \right]
\end{align*}

Cabbibo-Kobayashi-Maskawa matrix
\begin{align}
   \begin{split}
      V_\text{CKM} &= S^\dagger_u S_d \\
      V_\text{CKM}^\dagger &= S^\dagger_d S_u
   \end{split}
\end{align}
It is unitary and related to CP-violation.

How many parameters are there in $V_\text{CKM}$? $3 \times 3$ complex matrix in general has $18$-real parameter.

How many phases? $\bar{q} i \slashed{\partial} q$ not affected by (global) phase. Mass terms must change $q_L$ and $q_R$ by same phase.

First doublet
\begin{align}
   Q_{1L} = \begin{pmatrix}  u \\ V_{11}d + V_{12} s + V_{13} b \end{pmatrix}_L
\end{align}
$V_{11} = (V_\text{CKM})_{11}$, $V_{11} = R_{11} \euler^{i\delta}$, $R_{11} \in \R$. Use the notation $u = u' \euler^{i\delta}$ and $V = V' \euler^{i\delta}$
\begin{align*}
   Q_{1L} &= \euler^{i\delta_1} \begin{pmatrix} u' \\ R_{11}d + V'_{12} s + V'_{13} b \end{pmatrix} \\
   Q_{2L} &= \euler^{i\delta_2} \begin{pmatrix} c' \\ R_{21}d + V'_{22} s + V'_{23} b \end{pmatrix} \\
   Q_{3L} &= \euler^{i\delta_3} \begin{pmatrix} t' \\ R_{31}d + V'_{32} s + V'_{33} b \end{pmatrix}
\end{align*}

Overall phases of doublets $Q_{iL}$ don't affect anything. Still have freedom of phase shift in $s$ and $b$. Get rid of phase in $V'_{12}$ and $V'_{13}$
\begin{align*}
   V_{22}' \mapsto V''_{22} \quad V'_{23} \mapsto V''_{23}  \quad \dots
\end{align*}

Started with $18$ parameters, absorbed $5$ phases. It ends up with $13$ parameters. They are $9$ real parameter, $4$ phases. Normalization condition on $3$ guys indicates that all states have to orthogonal. Thus $6$ states. $13 - 3 - 6 = 4$ It has $4$ real parameter in $3 \times 3$ case.

Then $V_\text{CKM}$ is a orthogonal matrix multiplied with a phase. A orthogonal $3 \times 3$ matrix has $3$ real and $1$ phase parameters. This phase leads to CP-violation.

%%%%%%%%%%%%%%%%%%%%%%%%%%%%%%%%%%%%%%%%%%%%%%%%%%%%%%%%%%%%%%%%%
% Lecture date: 19-11-26
%%%%%%%%%%%%%%%%%%%%%%%%%%%%%%%%%%%%%%%%%%%%%%%%%%%%%%%%%%%%%%%%%
CKM matrix $n \times n$ cases. There are $2n^2$ (real) parameters. Unitarity of CKM matrix conditions, then $n^2$ parameters. $(2n-1)$ phases can be removed by a redefinition of quark phases $2 \cdot 3 - 1 = 5$ ($n=3$). 

$n \times n$ orthogonal matrix has $n(n-1)/2$ real parameters.
Number of phases
\begin{align*}
   n^2 - (2n-1) - n(n-1)/2 = n^2 - 2n + 1 - \frac{1}{2} n^2 + \frac{1}{2}n \\
   &= \frac{1}{2}n^2 - \frac{3}{2}n + 1 = \frac{1}{2} (n-1)(n-2)
\end{align*}

\begin{itemize}
   \item $n=2$ There is no phase
   \item $n=3$ $1$ phase \sm
   \item $n=4$ $3$ phases
\end{itemize}

The phases lead to CP-violation.
