\documentclass[oneside]{scrbook}
\KOMAoptions{fontsize=11pt, paper=a4}     
\KOMAoptions{DIV=13}                      

\usepackage[utf8]{inputenc}               
\usepackage[T1]{fontenc}                  
\usepackage[autostyle=true]{csquotes}     
\usepackage[varg]{txfonts}  			  %	Times-like fonts in support of mathematics
\usepackage{siunitx}   	  				  
\usepackage{enumitem}				      %	extra enumerate options

\renewcommand{\familydefault}{\rmdefault} % font to sans serif

%import external graphics and where to find these
\usepackage{graphicx}					  
\graphicspath{{figs/}}

\RequirePackage[backend=biber, style=numeric]{biblatex}
\addbibresource{refs.bib}

\usepackage{hyperref}
\RequirePackage[all]{hypcap}

\let\iint\relax
\let\iiint\relax
\let\iiiint\relax
\let\idotsint\relax
\usepackage{amsmath}
\usepackage{physics}
\usepackage{mathtools}
\usepackage{braket}
\usepackage{slashed} % feynman slash notation
\usepackage{simplewick} % wick contraction
\usepackage{tikz-feynman} % feynman diagrams 

% Externalizing plots
\usetikzlibrary{external}
\immediate\write18{mkdir -p tikz-figs}
\tikzexternalize[ prefix=tikz-figs/, 
                  mode=list and make,
                  system call={ lualatex \tikzexternalcheckshellescape -halt-on-error -interaction=batchmode -jobname="\image" "\texsource"  || rm "\image.pdf"},
]

%restart footnotes every page
\usepackage{perpage}
\MakePerPage{footnote}
%symbols for footnotes
\usepackage[symbol]{footmisc}
\renewcommand{\thefootnote}{\fnsymbol{footnote}} % footnote mark with special symbols

% toprule and etc.
\usepackage{booktabs}

%%%%%%%%%%%%%%%%%%%%%%%%%% NEW COMMAND SECTION %%%%%%%%%%%%%%%%%%%%

%define equal
\newcommand{\defeq}{\vcentcolon =} 
\newcommand{\eqdef}{= \vcentcolon}
\newcommand{\euler}{\mathrm{e}}

%Lagrange density
\newcommand{\lag}{\mathcal{L}} 
%Hamiltonian density
\newcommand{\ham}{\mathcal{H}}

%identity matrix
\usepackage{dsfont}
\newcommand{\id}{\mathds{1}}

\newcommand{\vecnab}{\pmb{\nabla}}
\newcommand{\vecx}{\pmb{x}}
\newcommand{\vecy}{\pmb{y}}
\newcommand{\veck}{\pmb{k}}
\newcommand{\vecp}{\pmb{k}}
\newcommand{\N}{\mathbb{N}}
\newcommand{\R}{\mathbb{R}}
\newcommand{\Co}{\mathbb{C}}
\newcommand{\M}{\mathcal{M}}
\newcommand{\sm}{Standard Model }
\newcommand{\diag}{\text{diag}}
\newcommand{\sgn}{\text{sgn}}
\newcommand{\gf}{\mathcal{G}}

\newcommand{\Uni}{\mathbf{U}}
\newcommand{\SU}{\mathbf{SU}}
\newcommand{\EM}{\text{EM}}

\newcommand{\UEM}{\Uni(1)_{\EM}}
\newcommand{\UY}{\Uni(1)_{\text{Y}}}
\newcommand{\SUL}{\mathbf{SU}(2)_\text{L}}
\newcommand{\SUC}{\mathbf{SU}(3)_\text{C}}

%%%%%%%%%%%%%%%%%%%%%%%%%%%%% SETTINGS %%%%%%%%%%%%%%%%%%%%%%%%%%%%%%%%%%%%

\numberwithin{equation}{section}

%%%%%%%%%%%%%%%%%%%%%%%%%%%%%%%%%%%%%%%%%%%%%%%%%%%%%%%%%%%%%%%%%%

\title{Theoretical particle physics}
\author{Chenhuan Wang}
\date{\today}
\begin{document}
\maketitle
\tableofcontents

\setcounter{chapter}{-1}
\chapter{Organisational}
Tutorials:
Thursday:   8-10, 10-12
Friday:     10-12, 13-15 

\paragraph{Exam} consists of four problems
\begin{itemize}
   \item first quickies
   \item 2nd-4th: similar in style to homework; two will be very close to homework 
\end{itemize}
One needs $50\%$ of points from homework. May hand in pairs.

\paragraph{Content of the lectures}
\begin{itemize}
   \item \sm of particle physics
   \item Electroweak sector
      \begin{itemize}
         \item gauge principle
         \item Higgs mechanism
         \item Yukawa interactions
         \item CP-violation
      \end{itemize}
\end{itemize}

\paragraph{Exercises}
\begin{itemize}
   \item go through basics of computing Feynman diagrams
   \item not to derive the formalism
   \item Lagrange $\rightarrow$ Feynman rules $\rightarrow$ amplitudes $\rightarrow$ cross section and decay rates (measured quantities)
\end{itemize}

\paragraph{Literature}
\begin{itemize}
   \item Halzen and Martin, Quarks and Leptons (a lot of basics of QCD)
   \item Cheng and Li (includes also quantum field theory topics CP-violation in \sm)
   \item Mark Thomson
   \item QFT basics
      \begin{itemize}
         \item Peskin and Schroeder
         \item M.Schwartz
         \item Ryder
      \end{itemize}
   \item Okun, Leptons and Quarks
\end{itemize}

\chapter{Introduction}
\section{\sm}
The \sm is the fundamental theory of nature. There are three interactions included
\begin{itemize}
   \item electromagnetic
   \item weak
   \item strong
   \item Higgs boson exchange
\end{itemize}
Electromagnetic and weak interactions can be unified into electroweak interactions. 

In the Sun, all these three interactions and gravity are present
\begin{itemize}
   \item Photons reaching us clearly indicating QED's presence.
   \item Neutrinos produced in weak interaction. Four protons to two protons and two neutrons (Helium). 
      Only weak interaction can change the colour of quarks. 
   \item Binding of Helium via strong interaction and binding energy released as kinetic energy.
   \item Gravity brings protons together and at high temperature to give Helium.
\end{itemize}

\paragraph{Gauge theories} (Lie groups algebras)
\begin{itemize}
   \item EM: $\UEM$, $\UY$
   \item weak: $\SUL$
   \item strong: $\SUC$
\end{itemize}

Forces in quantum theories involve exchange particles spin $1$ vector bosons
\begin{itemize}
   \item Electromagnetism with photon $\gamma$
   \item Weak interaction with $W^{\pm}$, $W^0$ and $Z^0$ (mixture of $W^0$ and hyper charge) (discovered at CERN)
   \item Strong interaction with $g^a, a=1,\dots,8$ gluons (discovered at DESY)
\end{itemize}

\paragraph{particles with spin $ \frac{1}{2} $} \hspace{0pt} \\
Leptons
\begin{align*}
   \begin{pmatrix} \nu_e \\ e^- \end{pmatrix}_L , e^-_R ;\quad
   \begin{pmatrix} \nu_\mu \\ \mu^- \end{pmatrix}_L, \mu_R^-; \quad
   \begin{pmatrix} \nu_\tau \\ \tau^- \end{pmatrix}_L, \tau_R^- 
\end{align*}
Quarks
\begin{align*}
   \begin{pmatrix} u \\ d \end{pmatrix}_L, u_R, d_R; \quad
   \begin{pmatrix} c \\ s\end{pmatrix}_L, c_R, s_R; \quad
   \begin{pmatrix} t \\ b\end{pmatrix}_L, t_R, b_R
\end{align*}

One complete generation
\begin{align*}
   \begin{pmatrix} \nu_e \\ e^- \end{pmatrix}_L , e^-_R ;\quad 
   \begin{pmatrix} u \\ d \end{pmatrix}_L, u_R, d_R; \quad
\end{align*}
To remove any one part, then gauge theory is inconsistent. It is known as "anomaly". 

\paragraph{Higgs boson} $h^0$, spin $0$ \\
In the \sm, Higgs bosons are described by complex scalar fields $\begin{pmatrix} H^+ \\ H^0 \end{pmatrix} $ . $h^0$ is the only fundamental scalar in nature, as far as we know.

\section{Energy scales}
\begin{itemize}
   \item Binding energy of atoms $1-10 \si{\eV}$
   \item Binding energy of nucleons $\approx 1 \si{\mega\eV}$
   \item No known binding energy in particle physics
   \item Protons and neutrons $\approx \Lambda_{QCD} \approx \mathcal{O} (100 \si{\mega\eV})$
\end{itemize}

Particles have masses
\begin{itemize}
   \item Electron $m_e = \SI{511}{\kilo\eV}$
   \item Muon $m_\mu = \SI{105}{\mega\eV}$
   \item Tau $m_\tau = \SI{1.7}{\giga\eV}$
   \item Neutrinos $m_\nu < \SI{1}{\eV}$ 
   \item Quarks\footnote{mass of proton mainly comes from dynamical effect "gluon"} 
      \begin{itemize}
         \item $m_u \approx \SI{3}{\mega\eV}$
         \item $m_d \approx \SI{5}{\mega\eV}$ 
         \item $\dots$
      \end{itemize}
   \item Photon $m_\gamma = 0$
   \item Gluon $m_g = 0$
   \item Higgs $m_{Higgs} \approx \SI{125}{\giga\eV}$
\end{itemize}

\paragraph{Colliders}
\begin{itemize}
   \item Large Electron-Positron Collider (LEP) operating 1989-2000: $\SI{91}{\giga\eV} - \SI{200}{\giga\eV}$
   \item Tevatron($p\bar{p}$-collider) operating 1983-2011: $\SI{800}{\giga\eV} - \SI{2}{\tera\eV}$
   \item LHC operating since 2010: $\SI{7}{\tera\eV} - \SI{13}{\tera\eV}$
\end{itemize}

\section{Natural units}
It is convenient to use the unit system, where we define
\begin{align}
   \hbar = c = 1, \\
   k_B = 1.
\end{align}
Everything is expressed in term of powers of energy.
\begin{align*}
   \SI{1}{\femto \m} = \SI{1e-15}{\m} = \SI{5}{\per\giga\eV}   
\end{align*}

\chapter{Lorentz Transformation}
\section{Introduction}
Metric (used for distance measuring) is defined as
\begin{align}
   g_{\mu\nu} = \begin{pmatrix} 1 & 0 & 0 & 0 \\ 0 & -1 & 0 & 0 \\ 0 & 0 & -1 & 0 \\ 0 & 0 & 0 & -1\end{pmatrix}.
\end{align}
In string and general relativity, people tend to use the signature $\diag(-, +, +, +)$.

By defining the four-momentum, the relativistic dispersion relation can be given
\begin{align}
   p &= (E, \pmb{p}), \\
   p^2 &= E^2 - \pmb{p}^2 = m^2.
\end{align}

Light is always light-like
\begin{align*}
   k^2 = t^2 - (x^2 + y^2 + z^2) = 0.
\end{align*}

Greek indices always go from $0$ to $3$
\begin{align}
   r^2 = g_{\mu\nu} r^\mu r^\nu = t^2 - \pmb{r}^2.
\end{align}

Distance between two spacetime point is defined via
\begin{align}
   \left| r_A - r_B \right|  = \sqrt{(r_A - r_B)\cdot(r_A - r_B)} = \sqrt{r_A^2 + r_B^2 - 2r_A\cdot r_B}.
\end{align}

\section{Lorentz Transformation}
\textit{Lorentz transformation} is a transformation between two inertial frames moving with constant velocity $\pmb{v}$ with respect to each other (boosts).
\begin{align*}
   x &= (x_0, x_1, x_2, x_3), \\
   x' &= (x'_0, x'_1, x'_2, x'_3), \\
   x_0 &=  t .
\end{align*}

We define (in SI unit)
\begin{align}
   \beta = \frac{v}{c} \quad \text{or} \quad
   \pmb{\beta} = \frac{\pmb{v}}{c}.
\end{align}

Coordinates in these two frames are related like
\begin{align*}
   x'_0 &= \gamma(x_0 - \beta x_1), \\
   x'_1 &= \gamma (x_1 - \beta x_0), \notag \\
   x'_2 &= x_2, \notag \\
   x'_3 &= x_3. \notag \label{math:lorTrafo}
\end{align*}
Inverse transformation can be achieved with $\beta \mapsto -\beta$

$\gamma$-factor is defined via
\begin{align}
   \gamma = \frac{1}{\sqrt{1-{\beta}^2}}.
\end{align}
Since the particles cannot travel faster than light, $|\pmb{\beta}| \leq 1$, then $\gamma \geq 1$.

Alternative parametrization using \underline{rapidity} $\zeta$
\begin{align*}
   \beta &= \tanh(\zeta), \quad \gamma = \cosh(\zeta), \\
   \gamma \beta &= \sinh(\zeta).
\end{align*}

Insert this into equation (\ref{math:lorTrafo})
\begin{align*}
   x'_0 = x_0 \cosh(\zeta) - x_1 \sinh(\zeta), \\
   x'_1 = x_0 \sinh(\zeta) - x_1 \cosh(\zeta).
\end{align*}

We can turn this into matrices
\begin{align}
   \begin{pmatrix} x_0 \\ x_1 \\ x_2 \\ x_3\end{pmatrix}
   &=
   \begin{pmatrix} \gamma & \gamma\beta & 0 & 0 \\ \gamma\beta & \gamma & 0 & 0 \\ 0 & 0 & 1 & 0 \\ 0 & 0 & 0 & 1\end{pmatrix}
   \cdot
   \begin{pmatrix} x'_0\\ x'_1 \\ x'_2 \\ x'_3\end{pmatrix}, \notag \\
   x &= \Lambda x'.
\end{align}

\section{Mathematical Properties of Lorentz Transformation}
Distance is invariant under Lorentz transformation
\begin{align}
   s^2 = x_0^2 - x_1^2 + x_2^2 + x_3^2 = x^2.
\end{align}

Lorentz transformation includes
\begin{itemize}
   \item Rotation and boosts
   \item Parity $\pmb{x} \mapsto -\pmb{x}$
   \item Time reversal $t \mapsto -t$
\end{itemize}
We can also expand it with translation. It then turns to Poincaré group.

\subsection{Tensors}
Define a function of original coordinates ($\alpha=0,1,2,3$)
\begin{align}
   x'^\alpha = x'^\alpha (x^0, x^1, x^2, x^3).
\end{align}

If $x'^\alpha$ transforms like
\begin{align}
   x'^\alpha = \frac{x'^\alpha}{x^\beta} x^\beta,
\end{align}
it is called \underline{contravariant}

Consider derivative $\frac{\partial}{\partial x'^\alpha}$
\begin{align}
   \frac{\partial f(x)}{\partial x'^\alpha} = \frac{\partial f(x)}{\partial x^\beta} \frac{\partial x^\beta}{\partial x'^\alpha}.
\end{align}
We can see the $x'$ is now in the denominator. The objects transformed like this are called \underline{covariant}.

Consider the following generic objects: $A'^\alpha$ contravariant vector
\begin{align}
   A'^\alpha = \frac{\partial x'^\alpha}{\partial x^\beta} A^\beta.
\end{align}

$B'_\alpha$ is covariant
\begin{align}
   B'_\alpha = \frac{\partial x^\beta}{\partial x'^\alpha} B_\beta.
\end{align}
Note $(x^0, x^1, x^2, x^3)$ is contravariant.

The field strength tensor 
\begin{align*}
  F'^{\alpha\beta} = \frac{\partial x'^\alpha}{\partial x^\gamma} \frac{\partial x'^\beta}{\partial x^\delta} F^{\gamma\delta},
\end{align*}
is contravariant rank 2.
Mixed is also allowed 
\begin{align*}
  H'^\alpha_\beta = \frac{\partial x'^\alpha}{\partial x^\gamma} \frac{x^\delta}{\partial x'^\beta} H^\delta_\gamma.
\end{align*}

\subsection{Inner or scalar product}
Lorentz transformation is an isometry with Minkowski scalar product
\begin{align*}
   B' \cdot A' &= B'_\alpha A'^\alpha, \\
               &= \left( \frac{\partial x^\beta}{\partial x'^\alpha} B_\beta \right) \left( \frac{\partial x'^\alpha}{\partial x^\gamma} A^\gamma \right), \\
               &= \frac{\partial x^\beta}{\partial x^\gamma} B_\beta A^\gamma, \\
               &= \delta^\beta_\gamma B_\beta A^\gamma, \\
               &= B \cdot A.
\end{align*}

Infinitesimal distance is
\begin{align}
   \dd{s}^2 &= \left( \dd{x^0} \right)^2 - \left( \dd{\pmb{x}} \right)^2 \notag, \\
            &= (g_{\alpha\beta} \dd{x^\alpha}) \dd{x^\beta} = \dd{x_\beta} \dd{x^{\beta}}.
\end{align}
Thus we can use metric tensor to lower index $\dd{x_\beta} = g_{\alpha\beta} \dd{x^{\alpha}}$

\begin{align*}
   A^\alpha &= \left(A^0, \pmb{A} \right), \\
   A_\alpha &= \left(A^0, -\pmb{A} \right).
\end{align*}

\section{Matrix Representation of Lorentz Transformation}
\subsection{General Properties}
We have 
\begin{align*}
   x = \begin{pmatrix} x^0 \\ x^1 \\ x^2 \\ x^3 \end{pmatrix}, \quad 
   gx =  \begin{pmatrix} x^0 \\ -x^1 \\ -x^2 \\ -x^3 \end{pmatrix}.
\end{align*}
Then $a\cdot b = (a, gb) = g_{\mu\nu} a^\mu b^\nu = (ga, b) = a^Tgb = (ga)^T b$

Lorentz transforming the coordinate
\begin{align}
   x'^\mu = \Lambda^\mu_{\;\nu} x^\nu \Leftrightarrow x' = \Lambda x.
\end{align}

\begin{align*}
   g_{\mu\nu} x^\mu x^\nu &= g_{\sigma \tau} x'^\sigma x'^\tau, \\
                          &= g_{\sigma \tau} \Lambda^\sigma_{\; \mu} x^\mu \Lambda^\tau_{\;\nu} x^\nu, \\
                          &= g_{\sigma \tau} \Lambda^\sigma_{\; \mu} \Lambda^\tau_{\;\nu}x^\mu  x^\nu.
\end{align*}

Then we have the \textit{defining} rule of Lorentz group
\begin{align}
   \begin{split}
    g_{\mu\nu} &= g_{\sigma \tau}  \Lambda^\sigma_{\; \mu} \Lambda^\tau_{\;\nu}, \\
   g &= \Lambda^T g \Lambda.
   \end{split}
\end{align}

\paragraph{Properties} of Lorentz transformation
\begin{itemize}
   \item $\left|\det(\Lambda) \right| = 1$
   \item $\left|\Lambda^0_{\;0} \right| \geq 1$
\end{itemize}

The orthochronous Lorentz transformations ()$\Lambda$ forms a group. \\

Parity does not form a group
      \begin{align}
         \Lambda_P = \text{diag}(1,-1,-1,-1)
      \end{align}

Time reversal 
\begin{align}
   \Lambda_T = \text{diag}(-1,+1,+1,+1)
\end{align}

There are four classes of Lorentz transformations depending on $\left(\sgn(\det(\Lambda)), \sgn(\Lambda^0_0) \right)$
\begin{itemize}
   \item $(+, +)$ $\Lambda$
   \item $(- ,-)$ $\Lambda_T \Lambda$ 
   \item $(-, +)$ $\Lambda_P \Lambda$
   \item $(+, -)$ $\Lambda_T\Lambda_P \Lambda$
\end{itemize}

Orthochronous $\Lambda$ has 6 parameters, $3$ for boosts and $3$ for rotations. $\Lambda^T g \Lambda = g$ is actually $16$ equations. All matrices here are symmetric. Thus $6$ of $16$ are redundant. There are $10$ independent equations.
$\Lambda$ has $16$ entries and it has $16-10=6$ free parameters.

\subsection{Explicit Construction}
We will restrict ourselves in \textit{orthochronous} Lorentz transformations. The exponential function is defines via Taylor expansion. With $L \in \R^{4 \times 4}$
\begin{align*}
   \Lambda = \euler^L = \exp(L).
\end{align*}

From linear algebra we know (because of the properties of determinant and trace)
\begin{align}
   \det(\Lambda) = \det(\euler^L) = \euler^{\tr(L)}.
\end{align}

Since $\det(\Lambda) = 1$, $\tr(L) = 0$,
$gL$ is anti-symmetric
\begin{align*}
   \Lambda^T g \Lambda &= g, \\
   g \Lambda^T g \Lambda &= \id_4, \\
   g \Lambda^T g &= \Lambda^{-1}, \\
   \exp(g L^T g) &= \Lambda^{-1} = \exp(-L), \\
   \Leftrightarrow g L^T g &= -L, \\
   \Leftrightarrow (gL)^T &= - gL.
\end{align*}

Thus
\begin{align*}
   L = \begin{pmatrix} 0 & L_{01} & L_{02} & L_{03} \\ L_{01} & 0 & L_{12} & L_{13} \\ L_{02} & L_{12} & 0 & L_{23} \\ L_{03} & L_{13} & L_{23} & 0\end{pmatrix}.
\end{align*}

Define 6 basis matrices $S_{1,2,3}$ and $K_{1,2,3}$
\begin{align*}
   &S_1 = \begin{pmatrix} 0&0&0&0 \\ 0&0&0&0 \\ 0&0&0&-1 \\ 0&0&1&0\end{pmatrix}, \quad 
   &&S_2 = \begin{pmatrix} 0&0&0&0 \\ 0&0&0&1 \\ 0&0&0&0 \\ 0&-1&0&0\end{pmatrix}, \quad
   &&&S_3 = \begin{pmatrix} 0&0&0&0 \\ 0&0&-1&0 \\ 0&1&0&0 \\ 0&0&0&0\end{pmatrix}, \\
   &K_1 = \begin{pmatrix} 0&1&0&0 \\ 1&0&0&0 \\ 0&0&0&0 \\ 0&0&0&0\end{pmatrix}, \quad 
   &&K_2 = \begin{pmatrix} 0&0&1&0 \\ 0&0&0&0 \\ 1&0&0&0 \\ 0&0&0&0\end{pmatrix}, \quad 
   &&&K_3 = \begin{pmatrix} 0&0&0&1 \\ 0&0&0&0 \\ 0&0&0&0 \\ 1&0&0&0\end{pmatrix}. \quad 
\end{align*}

$S_i$ is the generator of $3$-dimensional rotations and $K_i$ is the generator of $3$-dimensional boosts. \textcolor{red}{WHAT IS THIS?}
\begin{align*}
   \hat{\pmb n} \in \R^3, & \quad |\hat{ \pmb n}| = 1, \\
   \hat{\pmb n} \cdot \pmb{S} &= n_1S_1 + n_2 S_2 + n_3 S_3, \\
   (\hat{\pmb n}\cdot \pmb{S})^3 &= - \hat{\pmb n} \cdot \pmb{S}, \\
   (\hat{\pmb n}\cdot \pmb{K})^3 &= + \hat{\pmb n} \cdot \pmb{S}.
\end{align*}

In the end, with $\pmb{\omega}, \pmb{\zeta} \in \R^3$
\begin{align*}
   L &= -\pmb{\omega} \cdot \pmb{S} - \pmb{\zeta}\cdot \pmb{K}, \\
   \Lambda &= \exp(-\pmb{\omega} \cdot \pmb{S} - \pmb{\zeta}\cdot \pmb{K}).
\end{align*}

$\pmb{\omega}$ is the axis of rotation, $|\pmb{\omega}|$ is then the angle of rotation.
$\tanh{|\pmb{\zeta}|} = \beta$ and $\frac{\pmb{\zeta}}{|\pmb{\zeta}|}$ is the direction of boost.

We now will look at concrete examples
\begin{itemize}
   \item $\pmb{\zeta} = 0,\; \pmb{\omega} = w \hat{e}_z$
      \begin{align*}
         \Lambda = \begin{pmatrix} 1 & 0 & 0 & 0 \\ 0 & \cos{\omega} & \sin{\omega} & 0 \\ 0 & -\sin{\omega} & \cos{\omega} & 0 \\ 0 & 0 & 0 & 0\end{pmatrix}.
      \end{align*}
      Rotational angle is $\omega$.
   \item $\pmb{\omega} = 0, \; \pmb{\zeta} = \zeta \hat{e}_x$
      \begin{align*}
         \Lambda = \begin{pmatrix} \cosh{\zeta} & -\sinh{\zeta} & 0 & 0 \\ -\sinh{\zeta} & \cosh{\zeta} & 0 & 0 \\ 0 & 0 & 1 & 0 \\ 0 & 0 & 0 & 1\end{pmatrix}.
      \end{align*}
\end{itemize}

Pure general boost $\pmb{\zeta}$
\begin{align*}
   \Lambda &= \exp(-\pmb{\zeta} \cdot \pmb{K}), \\
   \pmb{\zeta} &= \frac{\pmb{\beta}}{|\pmb{\beta}|} \tanh^{-1} |\pmb{\beta}|, \quad \hat{\pmb \beta} = \frac{\pmb{\beta}}{|\pmb{\beta}|}, \\
   \Lambda &= \exp(-\hat{\pmb \beta} \cdot \pmb{K} \tanh^{-1}(\beta)).
\end{align*}

\subsection{Algebra of generators}
Consider the commutation algebra of $S_{i=1,2,3}$ and $K_{i=1,2,3}$
\begin{align}
   \left[ S_i, S_j \right] &= \epsilon_{ijk} S_k,  \\
   \left[ S_i, K_j \right] &= \epsilon_{ijk} K_k, \\
   \left[ K_i, K_j \right] &= - \epsilon_{ijk} S_k.
\end{align}
The last equation causes Thomas precession in atomic physics.

Choose a different basis
\begin{align}
   \pmb{S}_+ = \frac{1}{2} \left( \pmb{S} + i \pmb{K} \right), \\
   \pmb{S}_- = \frac{1}{2} \left( \pmb{S} - i \pmb{K} \right).
\end{align}
Then we can calculate the algebra
\begin{align}
   \left[ S_{+, i}, S_{+, j} \right] &= i \epsilon_{ijk} S_{+, k}, \\
   \left[ S_{-, i}, S_{-, j} \right] &= i \epsilon_{ijk} S_{-, k}, \\
   \left[ S_{+, i}, S_{-, j} \right] &= 0.
\end{align}
In other word, the algebras are decoupled. This familiar algebra is angular momentum algebra $\SU(2)$.

Classification by two numbers ($j_+$, $j_-$)
\begin{align*}
   j_+ = 0, \frac{1}{2}, 1, \dots, \\
   j_- = 0, \frac{1}{2}, 1, \dots. \\
\end{align*}
Dimension $=(2j_+ + 1) (2j_- + 1)$. 

A field is scalar field if $j_+ = j_- = 0$. One fundamental example of scalar field is Higgs boson. Other scalar particles are just bound states.

There are two possible states with spin $\frac{1}{2}$: $(j_+, j_-) = (\frac{1}{2}, 0)$ and $(0, \frac{1}{2})$. For  $(\frac{1}{2}, 0)$ it is
\begin{align*}
   \pmb{S}_+ &= \frac{1}{2} \pmb{\sigma}, \\
   \pmb{S}_- &= 0, \\
   \pmb{S} &= \frac{1}{2} \pmb{\sigma}, \\
   i\pmb{K} &=  \frac{1}{2} \pmb{\sigma} .
\end{align*}
For $(0, \frac{1}{2})$ there is "-".

So two types of spin $\frac{1}{2}$ fermions
\begin{align*}
   \left(\frac{1}{2}, 0\right) \rightarrow e^-_L, \\
   \left(0, \frac{1}{2}\right) \rightarrow e^-_R. \\
\end{align*}
They have different transformation law. W boson only to $e^-_L$ but photon couples to both. 

Under parity transformation $e^-_L \leftrightarrow e^-_R$. Both particles are needed for theory to be invariant under parity transformation, like EM and strong interactions.

\chapter{Relativistic Quantum Field Theory}
In non-relativistic quantum mechanics
\begin{align*}
   E &= \frac{\pmb{p}^2}{2m} \\
   E &\mapsto i\hbar \frac{\partial}{\partial t} \\
   \pmb{p} &\rightarrow -i\hbar \pmb{\nabla}
\end{align*}

After promoting the momentum and energy into operators in dispersion relation we have the Schrödinger equation
\begin{align}
   i \frac{\partial}{\partial t} \psi + \frac{1}{2m} \pmb{\nabla}^2 \psi = 0
\end{align}

Density of probability is defined via
\begin{align}
   \rho = |\psi|^2 = \psi \psi^*
\end{align}
It obeys the continuity equation
\begin{align}
   -\frac{\partial}{\partial t } \int_V \rho \dd{V} &= \int \pmb{j} \cdot \pmb{n} \dd{S} \notag\\
                                                    &= \int_V \pmb{\nabla} \cdot \pmb{j} \dd{V} \notag \\
   \Rightarrow \frac{\partial \rho}{\partial t} + \div \pmb{j} &= 0
\end{align}

Writing this explicitly
\begin{align}
   \frac{\partial \rho}{\partial t} &= \frac{\partial }{\partial t} \left( \psi \psi^* \right) \notag\\
                                    &= \psi \frac{\partial \psi^*}{\partial t} + \psi^* \frac{\partial \psi}{\partial t} \notag \\
                                    &= \frac{i}{2m} \left( \psi^* \pmb{\nabla}^2 \psi -  \psi \pmb{\nabla}^2 \psi \right) \notag \\
   \Rightarrow \pmb{j} &= - \frac{i}{2m} \left( \psi^* \pmb{\nabla}^2 \psi -  \psi \pmb{\nabla}^2 \psi \right)
\end{align}

If we have a plane wave state, as an example 
\begin{align*}
   \psi &= N \euler^{i\pmb{p}\cdot \pmb{x} - i Et} \\
   \pmb{j} &= \frac{\pmb{p}}{m} |N|^2
\end{align*}

\section{Relativistic wave equation}
Now we enter the relativistic regime
\begin{align*}
   E^2 &= \pmb{p}^2 + m^2 \\
   p^\mu &= (E, \pmb{p}) \quad p_\mu = (E, -\pmb{p}) \\
   p^2 &= m^2
\end{align*}

Promoting energy and momentum into operators
\begin{align*}
   p^\mu &\mapsto i \partial^\mu \\
   \partial_\mu \partial^\mu &= \frac{\partial ^2}{\partial^2 t} - \nabla^2
\end{align*}

We have then Klein-Gordon equation
\begin{align}
   (\partial_\mu \partial^\mu + m^2) \phi(\pmb{x}, t) = 0
\end{align}

The current in KG-theory is conserved as well
\begin{align}
   j^\mu &= (\rho, \pmb{j}) = i \left( \phi^* \partial^\mu \phi - \phi \partial^\mu \phi^* \right) \\
   \partial_\mu j^\mu &= 0
\end{align}

An example solution
\begin{align*}
   \phi = N \euler^{-ip\cdot x} \\
   j^\mu = 2 p^\mu |N|^2
\end{align*}

In terms of Lorentz transformation
\begin{align*}
   \rho \sim E
\end{align*}

Energies of particles
\begin{align*}
   E^2 = \pmb{p}^2 + m^2 \\
   E = \pm \sqrt{\pmb{p}^2 + m^2}
\end{align*}

It also implies negative probability
\begin{align*}
   E > 0 \mapsto \rho > 0 \\
   E < 0 \mapsto \rho < 0
\end{align*}

\section{Feynman-Stückelberg Interpretation of negative energy states}
"Electron" with $E, \pmb{p}$ and charge $-e$
\begin{align*}
   j^\mu_{e^-} = 2e|N|^2(E,\pmb{p})
\end{align*}
"Positron" with $E, \pmb{p}$ and charge $+e$
\begin{align*}
   j^\mu_{e^+} = 2e|N|^2(E,\pmb{p})
   = - 2e|N|^2(-E,-\pmb{p})
\end{align*}

We can think of $E<0$ solution as particle flying backwards in time or $E > 0$ anti-particle forwards in time.
\begin{figure}[ht]
   \centering
   \includegraphics[width=0.8\linewidth]{fs-interpretation/fs-interpretation.eps}
   \caption{scattering process; horizontal time-axis; in the second diagram a electron positron pair is produced}%
   \label{fig:}
\end{figure}

In a relativistic systems we need to remember following points
\begin{itemize}
   \item anti-particles
   \item particle numbers are not conserved
\end{itemize}

\section{Electrodynamics (spin $1$)}
Maxwell equations are
\begin{align}
   \pmb{E} &= -\vec{\nabla} \phi - \frac{\dd}{\dd{t}}{\pmb{A}} \\
   \pmb{B} &= \vec{\nabla} \times \pmb{A}\\
   \vec{\nabla} \times \pmb{E} &= -\frac{\dd}{\dd{t}}{\pmb{B}} \\
   \div \pmb{B} &= 0
\end{align}

Field strength tensor and four-potential
\begin{align}
   F_{\mu\nu} &= \partial_\mu A_\nu - \partial_\nu A_\mu \\
   A^\mu(x) &= (\phi, \pmb{A})
\end{align}
The fields can be calculated from it
\begin{align}
   E^i &= F^{0i} = \partial^i A^0 - \partial ^0 A^i \\
   B_i &= -\epsilon_{ijk} \partial^i A^k = -\epsilon_{0ijk}F^{jk}
\end{align}

Often it is useful to use the dual tensor
\begin{align}
   \tilde{F}_{\mu\nu} &= \epsilon_{\mu\nu\sigma\tau} F^{\sigma \tau} \\
   \partial^{\mu} \tilde{F}_{\mu\nu} &= 0
\end{align}
is the second set of maxwell equations.

The other set of two equations is 
\begin{align}
   \partial_\nu F^{\mu\nu} = 4\pi j^\mu
\end{align}

$\pmb{E}, \pmb{B}$ are observable, $\pmb{A}$ is not. $A^\mu$ is not uniquely fixed by $\pmb{E}$ and $\pmb{B}$. It has the following gauge symmetry
\begin{align}
   \tilde{A}_{\mu} = A_\mu + \partial_\mu \Lambda(\pmb{x}, t)
\end{align}

Use this transformation to get
\begin{align}
   \partial_\mu A^\mu = 0
\end{align}

Plugging it back then we have the relativistic wave equation
\begin{align}
   \partial_\mu \partial^\mu A^\nu = 0
\end{align}
it essentially is Klein-Gordon equation with mass $m=0$

$A^\mu$ is a vector with spin $1$
\begin{align*}
   (j_+, j_-) = \left( \frac{1}{2}, \frac{1}{2} \right)
\end{align*}
It implied it has two transverse degrees of freedom. It has spin $1$ properties: $+1$, $0$, $-1$, in which $0$ mode does not exist.

\section{Description of Fermions}
Original motivation for Dirac. He wants a linear equation in $E$ or $\frac{\partial}{\partial t}$
\begin{align*}
   p^\mu \mapsto i\partial^\mu
\end{align*}

Take the ansatz
\begin{align*}
   i\hbar \frac{\partial}{\partial t}\psi &= H \psi \\
   &= (\vec{\alpha}\cdot \pmb{p} + \beta m ) \psi
\end{align*}
but $\pmb{\alpha}$ and $\beta$ unknown. It still has to obey the relativistic energy relation
\begin{align*}
   A &= \left( \alpha_ip_i + \beta m \right) \left( \alpha_ip_i + \beta m \right) \\
     &\stackrel{!}{=} \pmb{p}^2 + m^2 \\
     &= \alpha_i \alpha_j p_i p_j + \beta^2m^2 + \alpha_i \beta p_i m + \beta \alpha_j p_j m
\end{align*}

From this we demand
\begin{align}
   \beta^2 = 1 \\
   \alpha_i^2 = 1 \\
   \alpha_i \alpha_j + \alpha_j \alpha_i = 0 \\
   \alpha_i \beta + \beta \alpha_i = 0
\end{align}
So $\alpha$ and $\beta$ are not just numbers, but (can be proven to be) hermitian traceless matrices with eigenvalue $\pm 1$.  In addition, it only exits in even dimensions.
Since $\alpha_i$ and $\beta$ are $4\times4$ matrices. $\psi$ has to be a 4-component spinor.

For parity conservation need $(\frac{1}{2}, 0) \bigoplus (0, \frac{1}{2})$
Thus 
\begin{align*}
   \begin{pmatrix} \begin{pmatrix} & \\ & \end{pmatrix}_{2\times2} & \\ & \begin{pmatrix} & \\ &  \end{pmatrix}_{2\times2}   \end{pmatrix}
\end{align*}

There are different sets of $\alpha_i, \beta$ which satisfy the conditions. They are called representations.

Dirac-Pauli representation
\begin{align}
   \alpha_i &= \begin{pmatrix} 0 & \sigma^i \\ \sigma^i &  0 \end{pmatrix} \\
   \beta &= \begin{pmatrix} \id_2 & 0 \\ 0 & -\id_2\end{pmatrix}
\end{align}
with $\sigma^i$ the Pauli matrices.

Weyl (chiral) representation
\begin{align}
   \alpha^i &= \begin{pmatrix} -\sigma^i & 0 \\ 0 & \sigma^i \end{pmatrix}  \\
   \beta &= \begin{pmatrix} 0 & \id_2 \\ \id_2 & 0 \end{pmatrix}
\end{align}
they are mainly used in high energy physics (E $\gg m$).

\subsection{Gamma Matrices}
We now define 4 gamma matrices $\gamma^\mu$, $\mu=0,1,2,3$
\begin{align}
   \gamma^\mu = \left(\beta, \beta \pmb{\alpha} \right)
\end{align}
Note that having an index does not make it Lorentz vector.

The Clifford algebra is defined as following
\begin{align}
   \left\{ \gamma^\mu, \gamma^\nu \right\} = \gamma^\mu \gamma^\nu + \gamma^\nu \gamma^\mu = 2 g^{\mu\nu}
\end{align}

In Dirac-Pauli representation
\begin{align}
   \gamma^0 &= \begin{pmatrix} \id & 0 \\ 0 & -\id \end{pmatrix} \\
   \gamma^i &= \begin{pmatrix} 0 & \sigma^i \\ -\sigma^i & 0 \end{pmatrix} \\
   \gamma^5 &= i \gamma^0 \gamma^1 \gamma^2 \gamma^3 = \begin{pmatrix} 0 & \id \\ \id & 0\end{pmatrix}
\end{align}

In Weyl representation 
\begin{align*}
   \gamma^0 \leftrightarrow \gamma^5
\end{align*}

Rewriting the Dirac equation using $\gamma$s
\begin{align}
   i \partial_t \psi &= \left( \pmb{\alpha} \cdot \pmb{p} + \beta m  \right) \psi \notag\\
   i \partial_t \psi &= -i \pmb{\alpha} \cdot \vec{\nabla} \psi + m \beta \psi \notag\\
   i \beta \partial_t \psi &= -i \beta \pmb{\alpha} \cdot \vec{\nabla} \psi + m \psi \notag\\
   \left(i\gamma^\mu \partial_\mu - m \right) \psi &= 0
\end{align}
Conventionally we use $\phi$ for spin $0$ particle and $A_\mu$ for spin $1$.

It is convenient to also have an equation for $\psi^\dagger$. First one can show $\gamma^\mu^\dagger = \gamma^0 \gamma^\mu \gamma^0$.
\begin{itemize}
   \item $\mu = 0$: $\gamma^0 = \beta$ and $\gamma^0^\dagger = \gamma^0 \gamma^0 \gamma^0$ $\Rightarrow \beta^2 = \id_4$
   \item $\gamma^\mu^\dagger = (\beta \alpha^k)^\dagger = (\alpha^k)^\dagger \beta^\dagger = \alpha^k \beta = \beta^2 \alpha^k \beta = \beta \gamma^k \beta = \gamma^0 \gamma^k \gamma^0$
\end{itemize}

\begin{align}
   i \gamma^0 \partial_0 \psi + i \gamma^k \partial_k \partial - m \psi = 0 \notag\\
   -i \partial_0 \psi^\dagger (\gamma^0)^\dagger - i(\partial_k \psi^\dagger) \gamma^k^\dagger - m \psi^\dagger = 0 \notag \\
   -i \partial_0 \psi^\dagger \gamma^0 - i(\partial_k \psi^\dagger) \gamma^0 \gamma^k \gamma^0 - m \psi^\dagger = 0 \notag \\
   \shortintertext{define $\bar{\psi} = \psi^\dagger \gamma^0$}
   -i \partial_0 \bar{\psi} \gamma^0 - i \partial_\mu \bar{\psi} \gamma^\mu - m \bar{\psi} = 0 \notag \\
   i(\partial_\mu \bar{\psi}) \gamma^\mu + m \bar{\psi} = 0
\end{align}

\subsection{Free Particle Solution to Dirac Equation}
\begin{align*}
   \left( i \gamma^\mu \partial_\mu - m \right) \psi &= 0 \\
   \shortintertext{multiplying $\gamma^\nu \partial_\nu$ from left}
   i\gamma^\mu \gamma^\nu \partial_\mu \partial_\nu \psi - m \gamma^\nu \partial_\nu \psi &= 0 \\
   i \gamma^\mu \gamma^\nu \partial_\mu \partial_\nu \psi + i m^2 \psi &= 0 \\
\end{align*}

\begin{align*}
   \gamma^\mu \gamma^\nu  &= \frac{1}{2} \left( \gamma^\mu \gamma^\nu + \gamma^\nu \gamma^\mu  \right) \\
                          &= \frac{1}{2} \left( \gamma^\mu \gamma^\nu - \gamma^\nu \gamma^\mu + 2g^{\mu\nu} \right) \\
                          &= \frac{1}{2} \left[ \gamma^\mu , \gamma^\nu \right] + g^{\mu\nu}
\end{align*}
The commutator is anti-symmetric and multiplying to symmetric tensor (derivatives) the term must vanish.

Each component of spinor satisfies the Klein-Gordon equation.
\begin{align}
   (\partial_\mu \partial^\mu + m^2) \psi_i = 0
\end{align}

Thus we can write the solution as plane-wave
\begin{align}
   \psi = u(\pmb{p}) \euler^{-ipx}
\end{align}
$u(\pmb{p})$ is also a 4-component object but as function $\pmb{p}$ not $\pmb{x}$

Insert in back into Dirac equation, then we have Dirac equation in momentum space
\begin{align}
   \left( \gamma^\mu p_\mu - m \right) u(\pmb{u}) = 0
\end{align}

Solution by considering Dirac-Pauli representation
\begin{align*}
   \left( \slashed{p} - m \right) u(\pmb{p}) = \begin{pmatrix} (E-m) \id & - \pmb{p} \cdot \pmb{\sigma} \\ \pmb{p} \cdot \pmb{\sigma} & -(E+m) \id \end{pmatrix} 
   \begin{pmatrix} u_A \\ u_B\end{pmatrix}
\end{align*}

$\pmb{p} = 0$ then $E = \pm m$

\begin{itemize}
   \item $E = +m$ Two solutions 
      \begin{align*}
         u_B = \begin{pmatrix} 1 \\ 0 \end{pmatrix} , \begin{pmatrix} 0 \\ 1\end{pmatrix}
      \end{align*}

   \item $E=-m$
      \begin{align*}
         u_A = \begin{pmatrix} 1 \\ 0\end{pmatrix}, \begin{pmatrix} 0 \\ 1\end{pmatrix}
      \end{align*}
\end{itemize}

$\pmb{p} = 0$
\begin{align}
   \pmb{\sigma} \cdot \pmb{p} u_B &= (E-m) u_A \\
   \pmb{\sigma} \cdot \pmb{p} u_A &= (E+m) u_B
\end{align}

\begin{itemize}
   \item $E>0$ 
      \begin{align*}
         \chi^{(1)} &= \begin{pmatrix} 1 \\ 0\end{pmatrix} \\
         \chi^{(2)} &= \begin{pmatrix} 0 \\ 1\end{pmatrix}
      \end{align*}
\end{itemize}

Ansatz $u_A^{(s)} = \chi^{(s)}$
\begin{align}
   u_B^{(s)} = \frac{\pmb{\sigma}\cdot\pmb{p}}{E + m} &\quad u_A^{(s)} = \frac{\pmb{\sigma}\cdot\pmb{p}}{E + m} \chi^{(s)} \notag\\
   u(\pmb{p}) &= N \begin{pmatrix} \chi^{(s)} \\ \frac{\pmb{\sigma}\cdot\pmb{p}}{E + m} \chi^{(s)}\end{pmatrix}
\end{align}

$E < 0$ and $u_B^{(s)} = \chi^{(s)}$
\begin{align}
   u(\pmb{p}) = N \begin{pmatrix} -\frac{\pmb{\sigma}\cdot\pmb{p}}{E + m} \chi^{(s)} \\ \chi^{(s)}\end{pmatrix}
\end{align}

One can show $u^{(r)}^\dagger u^{(s)} = N^2 \delta^{rs}$

Two fold degeneracy in each case. $ \begin{pmatrix} 1 \\ 0\end{pmatrix}$ and $ \begin{pmatrix} 0 \\ 1\end{pmatrix} $ for $E > 0$ and $E < 0$. There must be another observable which commutes with $H$ and $\pmb{p}$.
\begin{align*}
   H = \gamma^i p_i + \gamma^0 m
\end{align*}
\begin{align}
   \pmb{S} \cdot \hat{P} = \frac{1}{2} \begin{pmatrix} \pmb{\sigma}\cdot\pmb{p} & 0 \\ 0 & \pmb{\sigma}\cdot \hat{p}\end{pmatrix}
\end{align}

Helicity
\begin{align}
   \frac{1}{2} \pmb{\sigma} \cdot \hat{\hat{p}} &= \frac{1}{2} \begin{pmatrix} \hat{p}_z & \hat{p}_x + i\hat{p}_y \\ \hat{p}_x -i \hat{p}_y & -\hat{p}_z\end{pmatrix} \\
   \det(\pmb{\sigma} \cdot \hat{p}) &= - \hat{p}^2 = - 1
\end{align}
  
Determinant is the product of two eigenvalues, then
\begin{align*}
   \lambda_1 + \lambda_2 = 0 \\
   \lambda_1 \cdot \lambda_2 = 1 \\
   \lambda_{1,2} = \pm 1
\end{align*}

Antiparticle solution $u^{(3,4)}(-\pmb{p}) e^{-i(-p)x} = v^{(2,1)}$
\begin{align}
   \left( \slashed{p} + m \right)v(\pmb{p}) = 0
\end{align}

Normalization is
\begin{align}
   \int \rho \dd{V} = 2E \\
   N = \sqrt{E+m}
\end{align}

Completeness relation (spin sums)
\begin{align}
   \sum u^{(s)}(p) \bar{u}^{(s)}(p ) = \left( \slashed{p} + m \right) \\
   \sum v^{(s)}(p) \bar{v}^{(s)}(p ) = \left( \slashed{p} - m \right)
\end{align}

Define a projector projecting out positive and negative energy states
\begin{align}
   \Lambda_{\pm} = \frac{\pm \slashed{p} + m}{2m}
\end{align}

In Chiral (Weyl) representation
\begin{align}
   \left( \slashed{p} - m \right) u(\pmb{p}) = \begin{pmatrix} m & p \cdot \sigma \\ \vecp \cdot \bar{\sigma} & -m \end{pmatrix} \begin{pmatrix} u_L \\ u_R\end{pmatrix}
\end{align}
$\bar{\sigma} = (\sigma^0 , -\pmb{\sigma})$ and $\sigma^0 = \id_2$

Weyl equation
\begin{align}
   -m u_L + p\cdot \sigma u_R = 0 \\
   p \cdot \bar{\sigma} u_L - m u_R = 0
\end{align}
if $m=0$, the equations decouple.


%%%%%%%%%%%%%%%%%%%%%%%%%%%%%%%%%%%%%%%%%%%%%%%%%%%%%%%%%%%%%%%%%
% Lecture date: 19-11-05
%%%%%%%%%%%%%%%%%%%%%%%%%%%%%%%%%%%%%%%%%%%%%%%%%%%%%%%%%%%%%%%%%
\chapter{Standard Model}
We will be looking at the gauge group $\SU(2)_\text{L} \times \Uni(1)_\text{Y}$. This will be spontaneously broken into $\Uni(1)_{\text{EM}}$. Strong interaction will add $\SU(3)_\text{C}$ and it doesn't get spontaneously broken.

\section{Leptons}
Reintroduce fermions
\begin{align*}
   \psi &= \begin{pmatrix} \chi_\text{L} \\ \chi_\text{R} \end{pmatrix} \\
   \psi_\text{L} &= \begin{pmatrix} \chi_\text{L} \\ 0 \end{pmatrix}
\end{align*}
$\SU(2)_\text{L}$ only interacts with $\psi_\text{L}$.

In Chiral representation
\begin{align}
   \gamma^5 &= \begin{pmatrix} - \id_2 & 0 \\ 0 & \id_2 \end{pmatrix}
   \shortintertext{The projection operators}
   P_{L} &= \frac{1}{2} (\id_2 - \gamma^5) = \begin{pmatrix} \id_2 & 0 \\ 0 & 0\end{pmatrix} \\
   P_{R} &= \frac{1}{2} (\id_2 + \gamma^5) = \begin{pmatrix} 0 & 0 \\ 0 & \id_2 \end{pmatrix}
\end{align}
with $P_L^2 = P_L$ and $P_L + P_R = \id$.

$\chi_{\text{L}, \text{R}}$ two-component Weyl spinors.
\begin{align*}
   \left( \psi_\text{L} \right)^\dagger = (P_\text{L} \psi)^\dagger = \psi^\dagger P_\text{L}^\dagger = \psi^\dagger P_\text{L}\\
   \overline{\psi}_\text{L} = (\psi_L)^\dagger \gamma^0 = \psi^\dagger P_L \gamma^0 = \overline{\psi} P_R
\end{align*}

Introduce a doublet of left-handed particles
\begin{align*}
   L &= \begin{pmatrix} \nu_\text{L} \\ e^-_\text{L} \end{pmatrix} \\
   \nu_\text{L} &= (\psi_\nu)_\text{L} =  \begin{pmatrix} \chi_{\nu \text{L}} \\ 0 \end{pmatrix}    \\
   e_\text{L}^- &= (\psi_{e^-})_\text{L} = \begin{pmatrix} \chi_{e^- \text{L}} \\ 0 \end{pmatrix} \\
   e_\text{L} &= (\psi_e)_\text{L} = \frac{1}{2} \left( \id - \gamma_5 \right) \psi_e \\
   \shortintertext{There is no right-handed neutrino in this theory (singlet).}
   R &= e_\text{R} = P_\text{R} (\psi_e) = (\psi_e)_\text{R}
\end{align*}

Under $\Uni(1)_\text{Y}$ the hypercharges are define as
\begin{align}
   {Y}(L) &= -1 \\
   {Y}(R) &= -2
\end{align}
Here each component of $L$ has $Y=-1$. They are chosen so that 
\begin{align*}
   Q_{\text{EM}} = T^3_\text{L} + \frac{1}{2} Y
\end{align*}
with $T^3_\text{L} = \frac{1}{2} \tau^3$ a generator of $\SU(2)$.

\begin{align*}
   T^3_L = \begin{pmatrix} \frac{1}{2} & 0 \\ 0 & -\frac{1}{2} \end{pmatrix} \\
   T^3_L L  = \begin{pmatrix} \frac{1}{2} \nu_L \\ -\frac{1}{2} e^-_L\end{pmatrix}
\end{align*}

Take a look at the examples
\begin{align*}
   Q_{\text{EM}}(\nu_\text{L}) &= T^3_\text{L}(\nu_\text{L}) + \frac{1}{2} Y(\nu_\text{L}) = \frac{1}{2} - \frac{1}{2} = 0 \\
   Q_{\text{EM}}(e^-_\text{L}) &= T^3_\text{L}(e_\text{L}) + \frac{1}{2} Y(e_\text{L}) = -\frac{1}{2} - \frac{1}{2} = 0 \\
   Q_{\text{EM}}(e^-_\text{R}) &= T^3_\text{L}(e_\text{R}) + \frac{1}{2} Y(e_\text{R}) = 0 - 1 = -1
\end{align*}
$T_L$ only acts on left handed particles under $\SU(2)$
\begin{align*}
   \psi_{e_R} &= e^{i(0)} \psi_{e_R}  \\
   \comm{T^a}{Y} &= 0
\end{align*}

Normal kinetic term in Dirac theory
\begin{align*}
   \lag_\text{kin} &= \bar\psi i \gamma^\mu D_\mu (P_R + P_L) \psi \\
   \shortintertext{To split this using the anti-commutator of $\gamma^5$ and other gamma matrices, $\gamma^\mu P_L = P_R \gamma^\mu$}
                   &= \bar\psi_L i \gamma^\mu D_\mu \psi_L + \bar\psi_R i \gamma^\mu D_\mu \psi_R
\end{align*}

Thus kinetic terms for leptons are
\begin{align}
   \lag_{\text{leptons}}^{\text{kin}} &= \bar{R} i \gamma^\mu D'_\mu R + \bar{L} i \gamma^\mu D_\mu L  \\
   % \shortintertext{$D'_\mu$ only acts on $R$ and $Y(R) = -2$}
   \begin{split}
    D'_\mu &= \partial_\mu - \frac{ig'}{2} Y B_\mu  \\
          &= \partial_\mu + ig' B_\mu 
   \end{split} \\
   \begin{split}
    D_\mu &= \partial_\mu - \frac{ig'}{2}Y B_\mu - ig \frac{\tau^a}{2} W^a_\mu \\
         &= \partial_\mu + \frac{i}{2} g' B_\mu - ig \frac{\tau^a}{2} W_\mu^a
   \end{split}
\end{align}

%%%%%%%%%%%%%%%%%%%%%%%%%%%%%%%%%%%%%%%%%%%%%%%%%%%%%%%%%%%%%%%%%
% Lecture date: 19-11-11
%%%%%%%%%%%%%%%%%%%%%%%%%%%%%%%%%%%%%%%%%%%%%%%%%%%%%%%%%%%%%%%%%

In electromagnetic $\Uni(1)_\text{EM}$ the charge generator $Q$ with $Q(e^-) = -1$. Covariant derivative
\begin{align}
   D_\mu = \partial_\mu - ie Q A_\mu
\end{align}

In $\Uni(1)_\text{Y}$ the gauge boson $B_\mu$. Field strength tensor $F^\text{Y}_{\mu\nu} = \partial_\mu B_\mu - \partial_\nu B_\mu$. Charge generator $Y/2$. Covariant derivative 
\begin{align}
   D_\mu = \partial_\mu - \frac{ig'}{2} Y B_\mu
\end{align}

In Dirac Lagrangian there is a mass term $m\bar\psi \psi$.
\begin{align*}
   \lag_m &= m\bar\psi \psi = m \bar\psi (P_R + P_L) \psi \\
          &= m (\bar\psi P_R \psi + \bar\psi P_L \psi) \\
          &= m (\bar\psi P_R^2 \psi + \bar\psi P_L^2 \psi) \\
          &= m (\bar\psi_L \psi_R + \bar \psi_R \psi_L)
\end{align*}
$\bar\psi \psi$ mass term is called Dirac mass term. There is only left handed neutrino, so we cannot write Dirac mass term for neutrino.  Although $\psi_L$ and $\psi_R$ transform differently under Lorentz transformation, $\bar\psi \psi$ is still Lorentz invariant.

Recall that $\Uni(1)_\EM$ in QED $\psi \mapsto \psi'=\euler^{i\alpha(x)Q}\psi$. If $Q$ is same for left and right-handed components, $\bar\psi \psi$ is gauge invariant. If left and right have different hypercharges $Y$, then $\bar\psi_R \psi_L$ not $\Uni(1)_Y$ gauge invariant.

How about $\SU(2)_\text{L}$ gauge invariance of $\bar\psi_L \psi_R$? Under $\SU(2)$
\begin{align*}
   R &\mapsto R' = R \\
   L &\mapsto L' = \euler^{i\alpha_a(x) \tau^a /2} L
\end{align*}
So it's obvious that $\bar\psi_R \psi_L$ not $\SU(2)_\text{L}$ gauge invariant.

Any fermionic mass term vanishes, if we require $\SU(2)_\text{L} \times \Uni(1)_\text{Y}$ gauge invariance. Spontaneous symmetry breaking to let fermion gain mass.

\section{Add scalars}
Introduce a Higgs doublet
\begin{align}
   \Phi = \begin{pmatrix} \phi^+ \\ \phi^0 \end{pmatrix}
\end{align}
The superscripts denote the charge the field carries.

From previous section $Q_\EM = T^3_L + \frac{1}{2}Y $, so 
\begin{align*}
   Q_\EM(\phi^+) &= +1 =  \frac{1}{2} + \frac{1}{2} Y \Leftrightarrow Y=+1 \\
   Q_\EM(\phi^0) &= 0 = -\frac{1}{2} + \frac{1}{2} Y \Leftrightarrow Y=+1
\end{align*}
Together $Y(\Phi) = + 1$

Lagrangian
\begin{align}
   \begin{split}
    \lag_\text{scalar} &= (D_\mu \Phi)^\dagger (D^\mu \Phi) - V(\Phi^\dagger \Phi) \\
   D_\mu &= \partial_\mu - \frac{ig'}{2} B_\mu - \frac{ig}{2} \tau_i W^i_\mu \\
   V &= \mu^2 \Phi^\dagger \Phi + \lambda (\Phi^\dagger \Phi)^2 \\
   \Phi^\dagger &= \begin{pmatrix} \phi^+ \\ \phi^0 \end{pmatrix}^\dagger = \begin{pmatrix} \phi^+ & \phi^0\end{pmatrix}^* = \begin{pmatrix} \phi^- & (\phi^0)^* \end{pmatrix}
   \end{split}
\end{align}

\section{Coupling of Scalars and Fermions}
Recall that simple complex scalar $\phi^0 \in \Co$, $\SU(2)$ singlet. Under Lorentz transformations (per definition) $\phi^0(x) \mapsto \phi^0 (x)$. $\bar\psi \psi$ is also Lorentz invariant. It means $\phi^0\bar\psi \psi$ is Lorentz invariant. This type of interaction is called Yukawa interaction.
\begin{align*}
   \lag_\text{Yukawa} &= - y_e \left( \overline{R} \Phi^\dagger L + \overline{L} \Phi R \right) \\
   &\feynmandiagram[small, inline=(a.base), horizontal=a to x]{ a --[scalar] x --[fermion] f1, x -- [anti fermion] f2,};
\end{align*}

How to combine $\Phi$, $L$ and $R$? 
\begin{align*}
   Y(\Phi) &= +1  \\
   Y(L) &= -1 \\
   Y(R) &= -2 \\
   Y(\bar{L}) &= +1 
\end{align*}

Gauge transformation of interaction term 
\begin{align*}
   \phi \bar\psi \psi \mapsto \phi' \bar\psi' \psi' = \euler^{i\alpha(Q(\phi) + Q(\bar\psi) + Q(\psi))} \phi \bar\psi \psi
\end{align*}
Gauge invariance means the sum of hypercharges is zero. $\Phi \bar{L} R$ is $\Uni(1)_Y$ gauge invariant. 

What if we want $\SU(2)_\text{L} \times \Uni(1)_\text{Y}$ gauge invariant. 

$R$ is $\SU(2)$ invariant. $\Phi$ is $\underline{2}$ under $\SU(2)$. $L$ and $\bar L$ are $\underline{2}$ under $\SU(2)$. For $\SU(2)$ these two kinds of representations are the same $\bar{\underline{2}} = \underline{2}$.
\begin{align*}
\Phi \bar L: \underline{2} \otimes \bar{\underline{2}} = \underline{3} \oplus \underline{1}_A
\end{align*}

$\Phi = \begin{pmatrix} \phi^+ \\ \phi^0 \end{pmatrix} = \begin{pmatrix} \phi_1 \\ \phi_2 \end{pmatrix}$.
$\underline{1}$ is $\SU(2)$ singlet. Need antisymmetric combination of $\Phi \bar L$ to get $\SU(2)$ singlet.

%%%%%%%%%%%%%%%%%%%%%%%%%%%%%%%%%%%%%%%%%%%%%%%%%%%%%%%%%%%%%%%%%
% Lecture date: 19-11-12
%%%%%%%%%%%%%%%%%%%%%%%%%%%%%%%%%%%%%%%%%%%%%%%%%%%%%%%%%%%%%%%%%

In components
\begin{align*}
   \overline{L} \Phi R &= \begin{pmatrix} \overline{\nu}_L & \overline{e}_L^- \end{pmatrix} \begin{pmatrix} \phi^+ \\ \phi^0 \end{pmatrix} e_R \\
   &= \overline{\nu}_L \phi^+ e_R + \overline{e}_L^- \phi^0 e_R
   \shortintertext{here $\sum Y_i=0$}
   \overline{R} \Phi^+ L &= \overline{e}^-_R \begin{pmatrix} \phi^- & \left(\phi^0 \right)^*\end{pmatrix} \begin{pmatrix} \nu_L \\ e_L^-\end{pmatrix} \\
   &= \overline{e}_R^- \phi^- \nu_L + \overline{e}_R^- \left( \phi^0 \right)^* e_L^-
\end{align*}

The \sm does not contain $\nu_R$. But let's consider $\nu_R$ anyway
\begin{align*}
   Q_\EM &= T^3_L + \frac{1}{2} Y \\
   0 &= 0 + \frac{1}{2} Y(\nu_R)
\end{align*}
So it doesn't couple to $B_\mu$.

How about Dirac mass term?
\begin{align*}
   \overline{\nu}_L \nu_R + \overline{\nu}_R \nu_L
\end{align*}
It is not $\Uni(1)_\text{Y}$ or $\SU(2)_\text{L}$ invariant. Not a huge problem since it was the same for electron before we introduce Higgs mechanism.

Cannot write interaction like $\overline{L} \nu_R \Phi$ since the hypercharge is not zero. Instead
\begin{align*}
   \tilde{\Phi} &= i \tau_2 \Phi^* = \begin{pmatrix} \phi^0 \\ -\phi^- \end{pmatrix} \\
   Y(\tilde{\Phi}) &= -1 \\
   \lag_{\nu_L} &= \overline{L} \tilde{\Phi} \nu_R  + \overline{\nu}_R \tilde{\Phi}^\dagger L
   \shortintertext{Dirac mass term for neutrino. If $\nu_R$ exists, this is a possible mass term.}
   \overline{L} \tilde{\Phi} \nu_R &= \begin{pmatrix} \overline{\nu}_L & \overline{e}^-_L \end{pmatrix} \begin{pmatrix} (\phi^0)^* \\ -\phi^- \end{pmatrix} \nu_R  \\
   &= \overline{\nu} (\phi^0)^* \nu_R - \overline{e}_L^- \phi^- \nu_R
\end{align*} 

\section{Spontaneous Symmetry Breaking (Mass and Mixing of Gauge Bosons)}
In the early universe $\mu = \mu(T)$ and the symmetry got broken. Consider the case $\mu^2 < 0$.
\begin{align}
   \expval{\Phi} = \frac{1}{\sqrt{2}} \begin{pmatrix} 0 \\ v \end{pmatrix}
\end{align}
with $v \in \R$ and $[v] = 1$. Field $\phi^0$ has $Q_\EM = 0$. $Y(\phi^0) \neq 0$ breaks the $\Uni(1)_Y$ and $T_L^3(\phi^0) \neq 0$ breaks $\SU(2)_L$.

As before the field shift is 
\begin{align*}
   \Phi &= U^{-1}(\pmb\xi) \frac{1}{\sqrt{2}} \begin{pmatrix} 0 \\ v+\eta(x) \end{pmatrix} \\
   U(\pmb\xi) &= \exp(-i \pmb\xi \cdot \pmb\tau / (2v))
\end{align*}
$\eta(x), \pmb{\xi}(x) \in \R$

\begin{align}
   \Phi &\mapsto \Phi' = U(\xi) \Phi = \frac{1}{\sqrt{2}} \begin{pmatrix} 0 \\ v + \eta \end{pmatrix} \label{math:PhiTrafo}\\ 
   L &\mapsto L' = U(\pmb\xi) L \\
   W_\mu &\mapsto W'_\mu \\
   \pmb\tau \cdot \pmb W'_\mu &= U(\pmb\xi) \left[ \pmb \tau \cdot \pmb{W}_\mu - \frac{i}{g} U^{-1}\partial_\mu U \right]
\end{align}

Insert transformation (\ref{math:PhiTrafo}) into $\lag$ and see the physical interpretation. Yukawa coupling
\begin{align*}
   - y_e \overline{L} \Phi R &= - y_e \begin{pmatrix} \overline{\nu}_e & \overline{e}_L^- \end{pmatrix} \frac{1}{\sqrt{2}} \begin{pmatrix} 0 \\ v + \eta(x)\end{pmatrix} e_R \\
   &= - \frac{y_e}{\sqrt{2}} \overline{e}_L^- (v+\eta(x)) e_R
\end{align*}
Thus $m_e = y_e v / \sqrt{2}$. Know $m_e$ we can fix $y_e v$. It does not predict the mass of electron, but we can accommodate. But we do predict $m_\nu = 0$.

There is an extra term
\begin{align*}
   - \frac{y_e}{\sqrt{2}} \overline{e}_L^- e_R \eta(x)
\end{align*}
So the coupling of Higgs is direct proportional to mass of fermion. $y_e \sim 10^{-6}$, it is then not highly unlikely to observe Higgs decay into electrons at LHC. We have already seen Higgs decay into tau and bottom quarks.

\begin{align}
   \lag_\text{scalar} = \frac{1}{2} (\partial_\mu \eta) (\partial^\mu \eta) + V \left(\left(\frac{v+\eta}{\sqrt{2}}\right)^2 \right)
\end{align} 

Write 
\begin{align*}
   \Phi &= \frac{v + \eta}{\sqrt{2}} \xi \\ 
   \xi &= \begin{pmatrix} 0 \\ 1\end{pmatrix}
\end{align*}
Multiply $(D_\mu \Phi)^\dagger (D^\mu \Phi)$ out but focus only on terms without derivative
\begin{align*}
   D_\mu &= \partial_\mu - \frac{i}{2} g' B_\mu - ig \frac{\tau^i}{2} W_\mu^i \\
   \lag  &\subset \frac{(v+\eta)^2}{8} \chi^\dagger \left[ (g'B_\mu^\dagger + g \tau^i W^i _\mu) \cdot (g' B^\mu + g \tau^i W^{i \mu})\right] \chi \\
   \chi^\dagger \tau^3 \chi &= \begin{pmatrix} 0 & 1\end{pmatrix} \begin{pmatrix} 1 & 0 \\ 0 & -1 \end{pmatrix} \begin{pmatrix} 0 \\ 1\end{pmatrix} = -1 \\
   \chi^\dagger \tau^2 \chi &= \begin{pmatrix} 0 & 1\end{pmatrix} \begin{pmatrix} 0 & -i \\ i & 0 \end{pmatrix} \begin{pmatrix} 0 \\ 1 \end{pmatrix} = 0 \\
   \chi^\dagger \tau^1 \chi &= 0\\
\end{align*}
Focus on $v^2$ terms, they are bilinear in gauge boson fields $\rightarrow$ mass terms.
\begin{align}
   \lag \subset \frac{1}{8} v^2 \left[ (g'B_\mu - g W^3_\mu) (g' B^\mu-gW^{3 \mu}) + g^2 (W_\mu^1)^2 + g^2 (W_\mu^2)^2 \right]
\end{align}
with $M^2_{W^1} = M^2_{W^2} = {g^2 v^2}/ {8}$.

Using the familiar formula $Q = T_e + \frac{1}{2} Y$ for the $\SU(2)$ gauge boson $W^{1,2,3}$. They form a triplet $T_3 = \pm1, 0$. Doublet $T_3 = \pm \frac{1}{2}$. Then $W_3$ has zero electric charge. $B_\mu$ also has zero electric charge. Focus on the electric neutral part
\begin{align*}
   \lag &\subset \frac{v^2}{8} \left[  g'^2 B_\mu B^\mu + g^2 W_\mu^3 W^{\mu 3} + g'g B_\mu W^{\mu 3} + g'gW^3_\mu B^\mu \right]
   \shortintertext{It has $W_\mu B^\mu$ mixing terms. Rewritten in matrix form}
        &= \frac{v^2}{8} \begin{pmatrix} B_\mu & W_\mu^3 \end{pmatrix} \begin{pmatrix} g'^2 & gg' \\ gg' & g^2 \end{pmatrix} \begin{pmatrix} B^\mu \\ W^{3\mu}\end{pmatrix}
\end{align*}
The $2 \times 2$ matrix  is mass matrix squared. Want to diagonalize this. Eigenvalues are (mass)$^2$. For $2 \times 2$ matrix, 
\begin{align}
   \det &= \lambda_1 \cdot \lambda_2 \\
   \tr &= \lambda_1 + \lambda_2
\end{align}
Obviously here $\det = 0$ and $\tr = g^2 + g'^2$. 

Eigenvectors are
\begin{align}
   \begin{split}
      Z_\mu &= \frac{-g W_\mu^3 + g' B_\mu}{\sqrt{g^2 + g'^2}} \\
      A_\mu &= \frac{g B_\mu + g' W_\mu^3}{\sqrt{g^2 + g'^2}}
   \end{split}\label{math:AZ}
\end{align}

\begin{align*}
   m^2(A_\mu) &= 0  \\
   M_Z &= \frac{v}{2} \sqrt{g^2 + g'^2 }
\end{align*}
Remember real fields have mass term $\frac{1}{2} m^2 A^\mu A_\mu$.

Go back to $W^1_\mu$ and $W^2_\mu$. Define
\begin{align}
   W^\pm &= \frac{1}{\sqrt{2}} (W_\mu^1 \pm W_\mu^2) \\
   W^+_\mu W^{- \mu} &= \frac{1}{2} \left[ (W_\mu^1)^2 + (W_\mu^2)^2 \right]
\end{align}

$\pmb W$ is a triplet $(W^+, W^0,  W^-)^T $. The superscripts denote the electric charge again.

% Use $Q = T_3 + \frac{1}{2} Y$.
Mass term $\frac{1}{2} m^2 (W_\mu^+ W^{- \mu} + W^-_\mu W^{\mu +})$
\begin{align*}
   \frac{1}{2} M_W^2 &= \frac{1}{8} g^2 v^2 \\
   M_W &= \frac{1}{2} g v
\end{align*}

Go back to leptons
\begin{align*}
   \lag_{\text{kin}}^\text{leptons} &= \overline{R} i \gamma^\mu D'_\mu R + \overline{L} i \gamma^\mu D_\mu L \\
   D'_\mu &= \partial_\mu + i \frac{g'}{2} B_\mu \\
   D_\mu &= \partial_\mu + \frac{i}{2} g' B_\mu - ig \frac{\tau^i}{2} W_\mu^i
\end{align*}
Use equation~(\ref{math:AZ}) to rewrite $B_\mu = f(A_\mu, Z_\mu)$ and $W_\mu^3 = f'(A_\mu, Z_\mu)$.

Insert that into $D'_\mu$ and $D_\mu$ and then into $\lag_\text{kin}^\text{leptons}$. Also replace ${W'_\mu}^2 \mapsto W_\mu^{\pm}$
\begin{align*}
   \overline{L} i \gamma^\mu D_\mu L &= \overline{L} i \gamma^\mu  \left( \partial_\mu + \frac{i}{2} g' B_\mu - ig \frac{\tau^i}{2} W_\mu^i \right) L \\
   \tau_1 &= \begin{pmatrix} 0 & 1 \\ 1 & 0 \end{pmatrix}, \quad \tau_2 = \begin{pmatrix} 0 & -i \\ i & 0\end{pmatrix} \\
   \tau_1 W^1_\mu + \tau^2 W_\mu^2 &= \begin{pmatrix} 0 & W_\mu^1 - iW_\mu^2 \\ W_\mu + iW_\mu^2 & 0 \end{pmatrix}
\end{align*}

Charged current
\begin{align*}
   A &= \overline{L} \gamma^\mu \left( \tau^1 W^1_\mu + \tau^2 W_\mu^2 \right) L \\
     &= \frac{g}{2} \begin{pmatrix} \overline{\nu}_L & \overline{e}_L \end{pmatrix} \gamma^\mu \begin{pmatrix} 0 & \sqrt{2} W_\mu^+ \\ \sqrt{2} W_\mu^- & 0 \end{pmatrix} \begin{pmatrix} \nu_L \\ e_L \end{pmatrix} \\
   &= \frac{g}{2} \begin{pmatrix} \overline{\nu}_L & \overline{e}_L \end{pmatrix} \gamma^\mu \begin{pmatrix} \sqrt{2} W_\mu^+ e_L \\ \sqrt{2} W_\mu^- \nu_L\end{pmatrix} \\
   &= \frac{g}{\sqrt{2}} \left[ \overline{\nu}_L W_\mu^+ \gamma^\mu e_L + \overline{e}_L W^-_\mu \gamma^\mu \nu_L \right]
\end{align*}
Diagrammatically
\begin{align*}
   \feynmandiagram[layered layout, inline=(v.base), horizontal=v to f]{
      i1[particle=\(\nu\)] --[anti fermion] v, 
      i2[particle=\(e\)] --[fermion] v --[photon] f[particle=\(W\)],
   }; \sim ig \gamma^\mu P_L
\end{align*}

Neutrino-electron scattering has the cross section $\sim g^4 / M^4_W$
\begin{align*}
   \feynmandiagram[horizontal=v1 to v2]{
      i1[particle=\(\nu_e\)] --[fermion] v1 --[fermion] f1[particle=\(e^-\)],
      i2[particle=\(e^-\)] --[fermion] v2 --[fermion] f2[particle=\(\nu_e\)],
      v1 --[photon, edge label=\(W\)] v2
   };
\end{align*}

Fermi constant
\begin{align}
   G_F = \frac{\sqrt{2} g^2}{8 M^2_W}
\end{align}

$\lag_\text{kin}^\text{leptons}$ can be extended to muons. Then we can draw diagram for muon decay. Decay rate $\Gamma \sim G_F^2$. $\tau_\mu \sim \SI{2}{\micro\s}$. $G_F \sim \frac{10^{-5}}{m^2_\mu}$
\begin{align*}
   \feynmandiagram[horizontal=i1 to v1, medium, layered layout]{
      i1[particle=\(\mu\)] -- [fermion]v1,
      v1 -- [fermion] f3[particle=\(\nu_\mu\)],
      v1 -- [photon, edge label=\(W\)] v2,
      {[same layer] v2, f3},
      v2 --[fermion] f1[particle=\(e^-\)],
      v2 --[anti fermion] f2[particle=\(\bar\nu_e\)],
      {[same layer] f1, f2},
   };
\end{align*}

%%%%%%%%%%%%%%%%%%%%%%%%%%%%%%%%%%%%%%%%%%%%%%%%%%%%%%%%%%%%%%%%%
% Lecture date: 19-11-25
%%%%%%%%%%%%%%%%%%%%%%%%%%%%%%%%%%%%%%%%%%%%%%%%%%%%%%%%%%%%%%%%%
\chapter{Scattering Theory}
\section{Non-relativistic Perturbation Theory}
Quantum states are eigenfunctions of (bare) Hamiltonians. They should be (complete) orthonormal basis for the Hilbert space.
\begin{align*}
   H_0 \phi_n = E_n \phi_n \\
   \int \dd[3]{x} \phi^*_m \phi_n = \delta_{mn} \\
\end{align*}

Schrödinger equation 
\begin{align*}
   (H_0 + V) \psi = i \dv{\psi}{t} \label{math:sch}
\end{align*}

Make the ansatz that an arbitrary state is a superposition of of $\phi_n$
\begin{align}
   \psi = \sum a_n(t) \phi_n (\pmb{x}) \euler^{-iE_n t}
\end{align}
Insert it into Schrödinger equation \ref{math:sch}
\begin{align*}
   i\sum \dv{a_n}{t} \phi_n (\pmb x) \euler^{-iE_n t} &= \sum V(\pmb x, t) a_n (t) \phi_n(\pmb{x}) \euler^{-iE_n t}
   \shortintertext{multiply by $\phi^*_f(\pmb x)$ and $\int \dd[3]{x}$}
   \dv{a_f}{t} &= -i \sum_n a_n(t) \int \dd[3]{x} \phi^*_f V \phi_n \cdot \euler^{i(E_f - E_n)t }
\end{align*}

Assume this potential acts for a finite time $T$, between $[-T/2, T/2]$.

The initial state is $\ket{i}$ and eigenstate of $H_0$. 
\begin{align*}
   a_i (-T/2) &= 1 \\ 
   a_n(-T/2) &= 0 \;  \text{ with }n \neq i \\
\end{align*}
\begin{align}
   \dv{a_f}{t} &= -i  a_n(t) \int \dd[3]{x} \phi^*_f V \phi_i \cdot \euler^{i(E_f - E_i)t } \label{math:nonRelPertAf}\\
   \shortintertext{integrate}
   a_f (t) &= -i \eval{\int_{-T/2}^t \dd{t'} \int \dd[3]{x} \phi^*_f V \phi_i \euler^{i(E_f - E_i)t}}_{t=-T/2} \notag
\end{align}

\begin{align}
   T_{fi} &= a_f(T/2) \notag \\
   &= -i \int_{-T/2}^{T/2} \dd{t} \int \dd[3]{x} \left[ \phi_f(\pmb{x}) \euler^{-iE_f t} \right]^* V(\pmb x, t) \left[ \phi_i(\pmb x) \euler^{-iE_i t} \right] \notag \\
   T_{fi} &= -i \int \dd[4]{x} \phi^*_f (x) V(x) \phi_i (x)
\end{align}
for $a_f(t) \ll 1$.

Assume static potential $V(x) = V(\pmb x)$
\begin{align}
   \begin{split}
    T_{fi} &= -i V_{fi} \int^{+\infty}_{-\infty} \dd{t} \euler^{-i(E_f - E_i)t}\\
          &= -2\pi i V_{fi} \delta(E_f - E_i) \\
   V_{fi} &= \int \dd[3]{x} \phi_f^*(\pmb x) V(\pmb{x}) \phi_i(\pmb x)
   \end{split}
\end{align}

As $\Delta E \rightarrow 0$, $\Delta t = \infty$, so infinite time is needed to get from $\ket{i} \rightarrow \ket{f}$. Define transition probability per unit time
\begin{align}
   W &= \lim_{T \rightarrow \infty} \frac{|T_{fi}|^2}{T} \notag \\
     &= \lim_{T \rightarrow \infty} 2 \pi \frac{|V_{fi}|^2}{T} \delta(E_f - E_i) \int_{-T/2}^{T/2} \dd{t} \euler^{i(E_f - E_i)t} \notag\\
     &= \lim_{T \rightarrow \infty} 2 \pi \frac{|V_{fi}|^2}{T} \delta(E_f - E_i) T \notag\\
     &= 2\pi |V_{fi}|^2 \delta(E_f - E_i) \label{math:deltaSqaure}
\end{align}

It is only meaningful after integration over initial and final states. In collider initial state is prepared with fixed momentum $\pmb{p_e}$. So no integration over initial states.  For proton (hadron), it is a bit more complicated since the quarks are moving inside of proton.

Define the density of final states $\rho(E_f)$. \underline{Fermi's Golden rule}
\begin{align}
   \begin{split}
      W_{fi} &= 2 \pi \int \dd{E_f} \rho(E_f) |V_{fi}|^2 \delta(E_f - E_i) \\
   &= 2 \pi |V_{fi}|^2 \rho(E_i)
   \end{split}
\end{align}

The result can be improved by considering higher order scatterings, i.e.~scatter multiple times. For each order insert the result for $a_f(t)$  into formula (\ref{math:nonRelPertAf}) for the $a_{n \neq i} (t)$
\begin{align}
   \dv{a_f}{t} = - \sum a_n (t) \int \dd \phi^*_f V\phi_n \dd[3]{x} \euler^{i(E_f - E_i)t}
   \shortintertext{replace}
   a_n(t) = -i \int_{T/2}^{t}\dd{t'} \int \dd[3]{x'} \phi^*_n V \phi_i \euler^{i(E_n - E_i)t} 
\end{align}

Then
\begin{align*}
   \dv{a_f}{t} = -i \int \dd[3]{x} \phi^*_f V \phi_i \euler^{i(E_f - E_i)t} + (-i)^2 \sum_{n \neq i} \int_{-T/2}^{t} \dd{t'} \euler^{i (E_n - E_i)t'} V_{fn} \euler^{i(E_f - E_n)t}
\end{align*}

Correlation to $T_{fi}$
\begin{align*}
   T_{fi} &= \dots - \sum_{n \neq i} V_{fn} V_{ni} \int_{-\infty}^{+\infty} \dd{t} \euler^{i(E_f - E_n)t} \int_{-\infty}^{t} \dd{t'} \euler^{i(E_n - E_i)t}
   \shortintertext{regulate the last exponential by adding a small constant $\epsilon$}
   T_{fi} &= \dots - 2\pi i \sum_{n \neq i} \frac{V_{fn} V_{ni}}{E_i - E_n + i \epsilon} \delta(E_f - E_i)
\end{align*}
This is second order in perturbation theory including time dependence.

We can treat it like Feynman diagram with vertex  $V_{ni}$ and propagator $1/(E_i - E_n + i\epsilon)$. 

It is not manifestly covariant. Potential is assumed static and it defines the preferred reference frame. To go to relativistic formulation, consider other particle(s), which create electromagnetic potential for first particle to scatter in and vice versa.

\section{Scalar Electrodynamics}
\subsection{Complex scalar}
\begin{align}
   \lag = (\partial_mu \phi)^* (\partial^\mu \phi) - m^2 \phi^* \phi
\end{align}

It has the equation of motion
\begin{align}
   (\partial^2 + m^2) \phi = 0
\end{align}

To include electromagnetism consider generalized momentum 
\begin{align*}
   p^\mu &\mapsto p^\mu +_ e A^\mu  \\
   i \partial^\mu &\mapsto i \partial^\mu + e A^\mu
\end{align*}
Insert it into equation of motion
\begin{align}
   \begin{split}
    (\partial^2 + m^2) \phi &= - V\phi \\
   V &= -ie (\partial_\mu A^\mu + A^\mu \partial_mu) - e^2 A^2
   \end{split}
\end{align}

Fine structure constant $\alpha = e^2/(4\pi) \approx 1/137 \ll 1$. So in first order drop the $e^2$ term.
\begin{align}
   T_{fi} &= -i \int \dd[4]{x} \phi^*_f{x} V(x) \phi_i(x) \notag \\
          &= -i \int \dd[4]{x} \phi^*_f(x) ie \left[ \partial_\mu A^\mu + A^\mu \partial_\mu  \right] \phi_i (x) \notag
   \shortintertext{integrate by parts and drop surface terms}
          &= -i \int j_\mu^{fi} A^\mu \dd[4]{x} \label{math:TJA}
\end{align}
with the current
\begin{align}
   j_\mu^{fi} (x) = -ie \left[ \phi^*_f (\partial_\mu \phi_i) - (\partial_\mu \phi_f)^* \phi_i \right]
\end{align}
The current is conserved for free field $\partial_\mu j^\mu (x) = 0$

Write $\phi_i$ and $\phi_f$ as plane waves
\begin{align*}
   \phi_i (x) &= N_i \euler^{-ip_i \cdot x} \\
   j_\mu^{fi} &= -e N_i N_f (p_i + p_f) \euler^{i (p_f - p_i) \cdot x}
\end{align*}

\subsection{Scalar Charged Scattering}
Consider potential for particle 1 to scatter caused by particle 2 and vice versa.
\begin{align*}
   \feynmandiagram[vertical=v1 to v2]{
      pa[particle=\(p_A\)] -- [fermion] v1 -- [fermion] pc[particle=\(p_C\)],
      pb[particle=\(p_B\)] -- [fermion] v2 -- [fermion] pd[particle=\(p_D\)],
      v1 -- [photon, edge label=\(\gamma\)] v2,
   };
\end{align*}

What potential $A_\mu$ does $j_\mu^{(2)}$ create for particle 1 to scatter in inhomogeneous Maxwell equation
\begin{align*}
   \partial^2 A^\mu = j^{\mu(2)} = -e N_B N_D (p_D + p_B)^\mu \euler^{i(p_D - p_B)\cdot x}
\end{align*}

Also consider photons as plane waves
\begin{align}
   \partial^2 \euler^{iqx} = -q^2 \euler^{iqx} \notag
   \shortintertext{thus}
   A^\mu = - \frac{1}{q^2} j^{\mu (2)}
\end{align}
with $q = p_D - p_B$.

Insert into \ref{math:TJA} 
\begin{align*}
   T_{fi} &= -i \int j_\mu^{(1)} (x) \left(-\frac{1}{q^2} \right) j^{\mu (2)} (x) \dd[4]{x} \\
          &= -i e^2 N_A N_B N_C N_D \int \dd[4]{x} \euler^{i(p_c + p_D - p_A - p_B)} (p_C + p_A)_\mu (p_D + p_B)^\mu \frac{-1}{q^2} \\
          &= -i N_A N_B N_C N_D (2\pi)^4 \delta^{(4)} (p_C + p_D - p_A - p_B) \M
   \shortintertext{The invariant matrix element}
   -i\M &= ie (p_A + p_C)^\mu \left( -i \frac{g_{\mu\nu}}{q^2}  \right) ie (p_B + p_D)^\nu
\end{align*}
It is symmetric in $(A,C)$ and $(B,D)$.

\subsection{From Amplitude to Cross Section}
In the laboratory we measure cross sections. Need to fix norm, $\phi = N \euler^{-ipx}$. The number density is defined by
\begin{align}
   \rho = 2 E |N|^2
\end{align}
The infinitesimal number of particles $\rho \dd[3]{x}$ is Lorentz invariant. Norm is chosen as
\begin{align}
   \int \rho \dd{V} = 2E
\end{align}

We are interested in Transition rate per unit time and volume. 
\begin{align*}
   W_{fi} &= \frac{|T_{fi}|^2}{T \cdot V} \\
   T_{fi} &= -i N_A N_B N_C N_D (2\pi)^4 \delta^{(4)}(p_c + p_D - p_A - p_B) \M
\end{align*}

Same trick as before (\ref{math:deltaSqaure}) to get rid of one of $\delta$
\begin{align*}
   \int \dd[4]{x} &= T \cdot V  \\
   W_{fi} &= \frac{1}{V^4} (2\pi)^4 \delta^{(4)} (p_C + p_D - p_A - p_B) |\M|^2
\end{align*}

Definition of cross section
\begin{align}
   \sigma = \frac{W_{fi}}{\text{initial flux}} \cdot (\text{number of final states})
\end{align}
Imagine particles in a box
\begin{align*}
   (\text{number of final state particles}) = \frac{V \dd[3]{p}}{(2\pi)^3 2 E}
\end{align*}

Density of incoming particles A  $2E_A / V$. The box has length $\pmb{v}_A \cdot t$. Number of beam particles passing through unit area $A$ per unit time is $\pmb{v}_A \cdot t 2 E_A/(V t)$.  Particle B (target). Initial flux is $|\pmb{v}_A| \frac{2E_a}{v} \frac{2E_B}{V}$
\begin{align*}
   \dd{\sigma} &= \frac{V^2}{|\pmb{v}_A| 2 E_A 2 E_B} \frac{1}{V^4} |M|^2 \frac{(2\pi)^4}{(2\pi)^4} \delta^{(4)} (p_c + p_D - p_A - p_B) \frac{\dd[3]{p_C}}{2\pi} \frac{\dd[3]{p_D}}{2\pi} V^2 \\
               &= \frac{|\M|^2}{F} \dd{Q}
\end{align*}
with $F = |\pmb{v}_A| 2 E_A 2 E_B$ and $\dd{Q} = (2\pi)^4 \delta^{(4)}(p_C + p_D - p_A - p_B) \frac{\dd[3]{p_C}}{(2\pi)^3 2E_C} \frac{\dd[3]{p_D}}{(2\pi)^3 2 E_D}$and $B$ at rest.

If both are moving $F = |\pmb{v}_A - \pmb{v}_B| 2 E_A 2E_B = 4 \sqrt{(p_A \cdot p_B)^2 - m_A^2 m_B^2 }$.

In centre of mass system
\begin{align}
   \dv{\sigma}{\Omega} = \frac{1}{64 \pi^2 s } \frac{p_f}{p_i} |M|^2 
\end{align}

For $e \mu$ scattering 
\begin{align*}
   \dv{\sigma}{\Omega} = \frac{\alpha^2}{4s} \left( \frac{3 + \cos(\theta)}{1 - \cos(\theta)} \right)
\end{align*}
with $\theta$ angle between $\pmb{p}_A$ and $\pmb{p}_C$.

Also it includes decays, for example $n \rightarrow p e^- \bar{\nu}_e$.

%%%%%%%%%%%%%%%%%%%%%%%%%%%%%%%%%%%%%%%%%%%%%%%%%%%%%%%%%%%%%%%%%
% Lecture date: 19-11-26
%%%%%%%%%%%%%%%%%%%%%%%%%%%%%%%%%%%%%%%%%%%%%%%%%%%%%%%%%%%%%%%%%
\paragraph{Decay versus cross section}
Examples are $\pi^- \rightarrow \mu^- \bar{\nu}_\mu$, $t \rightarrow W^+ b$

\begin{align}
   \dd{\Gamma} &= \frac{1}{2E_A} |\M|^2 \frac{\dd[3]p_1}{(2\pi)^3 2 E_1}\frac{\dd[3]p_2}{(2\pi)^3 2 E_2} \\
   \Gamma &= \int \dd{\Gamma}
\end{align}

Pion for instance has different decay modes.
\begin{align}
   \Gamma_\text{tot} = \Gamma(\pi \rightarrow \mu \nu_\mu) + \Gamma(\pi \rightarrow e \bar{\nu}_e)
\end{align}

Radioactive decay 
\begin{align}
   N_A (t) = N(0) \euler^{-\Gamma_{\text{tot}}t}
\end{align}

Lifetime is defined as 
\begin{align}
   \tau_A = 1/\Gamma_\text{tot}(A)
\end{align}

\section{Electron Electron scattering}
\begin{align*}
   \feynmandiagram[vertical=v1 to v2]{
      i1[particle=\(e^-\)] --[momentum'=\(p_A\)] v1 --[momentum'=\(p_C\)] f1[particle=\(e^-\)],
      i2[particle=\(e^-\)] --[momentum=\(p_B\)] v2 --[momentum=\(p_D\)] f2[particle=\(e^-\)],
      v1 --[photon, edge label=\(\gamma\)] v2,
   };
   \quad
   \begin{tikzpicture}
      \begin{feynman}
      \diagram [vertical=a to b] {
         i1 [particle=\(e^{-}\)] --[momentum'=\(p_A\)]  a -- [draw=none] f1 [particle=\(e^{-}\), label=0:\(p_C\)],
         i2 [particle=\(e^{-}\)] --[momentum=\(p_B\)]  b -- [draw=none] f2 [particle=\(e^{-}\), label=0:\(p_D\)],
         a -- [photon, edge label'=\(\gamma\)] b,
      };
      \diagram* {
         (a) --  (f2),
         (b) --  (f1),
      };
      \end{feynman}
   \end{tikzpicture}
\end{align*}

One cannot differentiate electron from electron. We have to add both diagrams before squaring
\begin{align*}
-iM_{e^- e^-} = -i e^2 \left[ - \frac{(p_A + p_C)_\mu \cdot (p_B + p_D)^\mu}{(p_D - p_B)^2} - \frac{(p_A + p_D)_\mu \cdot (p_B + p_C)^\mu}{(p_C - p_B)^2}\right]
\end{align*}

\section{Electron Positron scattering}
\begin{align*}
      \feynmandiagram[vertical=v1 to v2]{
      i1[particle=\(e^-\)] --[momentum'=\(p_A\)] v1 --[momentum'=\(p_C\)] f1[particle=\(e^-\)],
      i2[particle=\(e^+\)] --[momentum=\(p_B\)] v2 --[momentum=\(p_D\)] f2[particle=\(e^+\)],
      v1 --[photon, edge label=\(\gamma\)] v2,
   };
   \quad
   \feynmandiagram[horizontal=v1 to v2]{
      i1[particle=\(e^-\)] --[momentum'=\(p_B\)] v1 --[reversed momentum'=\(p_A\)] f1[particle=\(e^-\)],
      i2[particle=\(e^-\)] --[reversed momentum=\(p_D\)] v2 --[momentum=\(p_C\)] f2[particle=\(e^-\)],
      v1 --[photon, edge label=\(\gamma\)] v2,
   };
\end{align*}

\section{Mandelstam Variables}
Only for two-to-two scattering
\begin{align}
   s &= (p_A + p_B)^2 \\
   u &= (p_A - p_C)^2 \\
   t &= (p_A - p-D)^2 
\end{align}

Diagrams are often classified by the momentum of propagator. Mandelstam variables $s$, $t$ and $u$ are not independent.
\begin{align}
   s + u + t = \sum m_i^2
\end{align}

One can set the reference frame such that
\begin{align*}
   p_A &= (E,0,0,E) \\
   p_B &= (E,0,0,-E) \\
   p_C &= (E,0,E \sin(\theta), E\cos(\theta)) \\
   p_D &= (E, 0, -E \sin(\theta), -E\cos(\theta))  
\end{align*}
Then 
\begin{align}
   s &= 4 E^2 \\
   t &= -\frac{s}{2}(1-\cos(\theta)) \\
   u &= -\frac{s}{2} (1+\cos(\theta))
\end{align}

\section{Origin of the propagator}
Non-relativistically 
\begin{align}
   T_{fi}^{(2)} = -i \sum_{n \neq i} V_{fn} \frac{1}{E_i - E_n} V_{ni} 2\pi \delta(E_f - E_i)
\end{align}

Electron positron scattering
each Feynman diagram is the sum of all possible time orderings in old fashioned perturbation theory. 

Two times orderings for s-channel diagram
\begin{align*}
   \includegraphics[width=0.8\linewidth]{propTime/1.eps}
\end{align*}
(one can understand the second diagram by saying that an amount of energy is borrowed to produce particle pair.)

\begin{align*}
   \M &\sim V_{fn} \frac{1}{E_i - E\gamma} V_{ni} + V_{fn} \frac{1}{E_i - 2E_i - E_\gamma} V_{ni} \\
      &= V_{fn} \frac{2E_\gamma}{E_i^2 - E_\gamma^2} V_{ni}
\end{align*}

\begin{align*}
   (p_A + p_B) ^2 &= E_i^2 - (\pmb{p}_A + \pmb{p}_B)^2  \\
   E_\gamma^2 &= \pmb{p}^2 + m_\gamma^2 \\
   p &= p_A + p_B \\
   \frac{1}{E_i^2 - E_\gamma^2} &= \frac{1}{(p_A + p_B)^2 - m_\gamma^2} = \frac{1}{q^2}
\end{align*}

%%%%%%%%%%%%%%%%%%%%%%%%%%%%%%%%%%%%%%%%%%%%%%%%%%%%%%%%%%%%%%%%%
% Lecture date: 19-12-09
%%%%%%%%%%%%%%%%%%%%%%%%%%%%%%%%%%%%%%%%%%%%%%%%%%%%%%%%%%%%%%%%%
\chapter{Electroweak Processes}
We will consider $\SU(2)_\text{L} \times \Uni(1)_\text{Y}$ theory.

\section{Muon decay}
$V-A$ interaction

\begin{align*}
   \feynmandiagram[horizontal=i1 to v1, medium, layered layout]{
      i1[particle=\(\mu\)] --[momentum=\(p\), fermion]v1,
      v1 --[momentum=\(k_1\), fermion] f3[particle=\(\nu_\mu\)],
      v1 -- [photon, edge label=\(W\)] v2,
      {[same layer] v2, f3},
      v2 --[fermion, momentum=\(k_2\)] f1[particle=\(e^-\)],
      v2 --[anti fermion, momentum=\(k_3\)] f2[particle=\(\bar\nu_e\)],
      {[same layer] f1, f2},
   };
\end{align*}

$q = k_2 + k_3 = p - k_1$
\begin{align*}
   i \M = \overline{u}(k_1)  \frac{-ig}{\sqrt{2}} \gamma^\mu P_L u(p) \frac{-ig_{\mu\nu}}{q^2 - M_W^2} \overline{u}(k_2) \frac{-ig}{\sqrt{2}} \gamma^\nu P_L v(k_3)
\end{align*}

In spinor space, Dirac equation 
\begin{align}
   (\slashed{p} - m) u(p) = 0.
\end{align}

Propagator $M_W \sim \SI{80}{\giga\eV}$ and $m_\mu = \SI{100}{\mega\eV}$, i.e.~$p^2 \ll M^2$. So $4$-Fermi approximation is obtained.
\begin{align*}
   \frac{1}{q^2 - M_W^2} = - \frac{1}{M_W^2} \frac{1}{1 - q^2 / M_W^2} \approx -\frac{1}{M_W^2}
\end{align*}
Then the coupling is effectively
\begin{align}
   G_F = \frac{\sqrt{2} g^2}{8 M_W^2}
\end{align}
Note that when for example top and bottom quarks are present, this approximation is not valid any more.

Proceed with $m_e = m_{\nu_e} = m_{\nu_\mu} = 0$
\begin{align*}
   \M &= -\frac{4G_F}{\sqrt{2}} \overline{u}(k_1) \gamma^\mu P_L u(p) \overline{u}(k_2) \gamma_\mu P_L v(k_3) \\
   \frac{1}{2} \sum_{\text{spins}} |\M|^2 &= \frac{16 G_F^2}{2 \cdot 2} \overline{u}(k_1) \gamma^\mu P_L u(p) \overline{u}(p) P_R \gamma^\nu u(k_1) \overline{u}(k_2) \gamma_\mu P_L v(k_3) \overline{v}(k_3) P_R \gamma_\nu u(k_2) \\
                                          &= 4 G_F^2 \tr[\slashed{k}_1 \gamma^\mu P_L (\slashed{p} + m_\mu) P_R \gamma^\nu] \cdot \tr[\slashed{k}_2 \gamma_\mu P_L \slashed{k}_3 P_R \gamma_\nu]
                                          \shortintertext{Term with $m_\mu$ vanishes, since $P_L P_R = 0$}
\end{align*}

Computing the two traces
\begin{align*}
   \tr[\slashed{k}_1 \gamma^\mu \slashed{p}\gamma^\nu P_L] &= \frac{4}{2}  \left( k_1^\mu p^\nu - k_1 p g^{\mu\nu} + k_1^\nu p^\mu \right) - \frac{4i}{2} \epsilon_{\alpha\mu\beta \nu} {k_1}_\alpha p_\beta  \\
   \tr[\slashed{k}_2 \gamma_\mu \slashed{k}_3 \gamma_\nu P_L] &= \frac{4}{2}  \left( (k_2)_\mu (k_3)_\nu - k_2 k_3 g_{\mu\nu} + {k_2}_\nu {k_3}_\mu \right) + \frac{4i}{2} \epsilon \epsilon_{\alpha \mu \beta \nu} k_2^\alpha k_3^\beta
\end{align*}
Using the fact that $\epsilon$ is totally anti-symmetric tensor, so the terms with one $\epsilon$ sum to zero. Note further
\begin{align}
   \epsilon^{\mu\nu\rho\sigma} \epsilon_{\mu\nu}^{\quad \rho'\sigma'} = -2 (g^{\rho \rho'}g^{\sigma \sigma'} - g^{\rho\sigma'} g^{\rho'\sigma})
\end{align}

In order to use this formula
\begin{align*}
   &\sigma^{\alpha \mu \beta \nu} \sigma_{\alpha' \mu \beta' \nu} \\
   &= \epsilon^{\mu \nu \alpha \beta} \epsilon^{\quad \alpha'' \beta''}_{\mu\nu} g_{\alpha' \alpha ''} g_{\beta ' \beta''} \\
   &= (-2) \left[ g^{\alpha \alpha''} g^{\beta \beta''} - g^{\alpha \beta''} g^{\beta \alpha''}\right] g_{\alpha' \alpha''} g_{\beta' \beta''} \\
   &= (-2) \left[ \delta^{\alpha}_{\alpha'} \delta^{\beta}_{\beta'} - \delta^{\alpha}_{\beta'} \delta^{\beta}_{\alpha'}  \right]
\end{align*}

Terms without $\epsilon$
\begin{align*}
   &4 (k_1^\mu p^\nu + k_1^\nu p^\mu - g^{\mu\nu} k_1 \cdot p) (k_{2 \mu} k_{3 \nu} + k_{2\nu} k_{3\mu} - k_2 \cdot k_3 g_{\mu\nu}) \\
   &= \dots = 4 \left[ 2 (k_1 \cdot k_2) (p\cdot k_3) + 2 (k_1\cdot k_3)( p \cdot k_2) \right] \\
   &= 8 \left[ (k_1 \cdot k_2)( p \cdot k_3) + (k_1 \cdot k_3) (p \cdot k_2) \right]
\end{align*}

Then
\begin{align*}
   \frac{1}{2} \sum_{\text{spins}} |\M|^2 = 64 G_F^2 (k_1 \cdot k_2)( p\cdot k_3)
\end{align*}
Recall
\begin{align}
   \begin{split}
   \dd{\Gamma} &= \frac{1}{2E} \overline{|\M|^2} \dd{Q} \\
   \dd{Q} &= \frac{\dd[3]{k_1}}{(2\pi)^3 2 E_{k_1}} \frac{\dd[3]{k_2}}{(2\pi)^3 2 E_{k_2}} \frac{\dd[3]{k_3}}{(2\pi)^3 2 E_{k_3}} (2\pi)^{4} \delta^{(4)} (p-k_1 -k_2 - k_3)
   \end{split}
\end{align}

One can prove (In \sm neutrino has no mass.)
\begin{align}
   \frac{\dd[3]{k_1}}{(2\pi)^3 2 E_{k_1}} = \int \dd[4]{k_1} \theta(E_1) \delta(k_1^2)
\end{align}

Use identity of Delta function involving composite function.
\begin{align*}
   \frac{\dd[3]{k_1}}{(2\pi)^3 2 E_{k_1}} \delta^{(4)}(p- k_1 -k_2 - k_3 ) = \int \dd[4]{k_1} \theta(E_1) \delta(k_1^2) \delta^{(4)}(p- k_1 -k_2 - k_3 ) \\
   = \theta(k_1)\delta^{k_1^2}
\end{align*}

Thus 
\begin{align*}
   \dd{Q} = \frac{1}{(2\pi)^5} \frac{\dd[3]{k_2}}{2E_{k_2}} \frac{\dd[3]{k_3}}{2E_{k_3}} \theta(E - E_{k_2} - E_{k_3}) \delta((p-k_2-k_3)^2)
\end{align*}

In rest frame of muon 
$p \cdot k_3 = m_\mu \cdot E_3$
\begin{align*}
   (k_1 + k_2)^2 &\approx 2 k_1 \cdot k_2 \\
   k_1 \cdot k_2 &\approx \frac{1}{2} (k_1 + k_2)^2 \\
                 &= \frac{1}{2} (p-k_3)^2 \\
                  &= \frac{1}{2} \left[ p^2 - 2 p\cdot k_3 + k_3^2 \right] \\
                  &= \frac{1}{2} [m_\mu^2 - 2 m_\mu E_3]
\end{align*}

\begin{align*}
   \frac{1}{2} \overline{|\M|^2} &= 64 G_F^2 k_1\cdot k_2 p \cdot k_3 \\
   &= 64 G_F^2 m_\mu E_3 [m_\mu^2 - 2 m_\mu E_3]
\end{align*}
Now we have computed $|\M|^2$, the actual difficulties are in phase space integral. For $n$-particle phase space, often use Monte-Carlo integration technique.

With $\theta$ the angle between electron and electron neutrino ($k_2$ and $k_3$)
\begin{align*}
   (p-k_2 - k_3)^2 = p^2 + k_2^2 + k_3^2 - 2 p \cdot k_2 -2 p\cdot k_3 + 2k_2 \cdot k_3 \\
   = m_\mu^2 - 2 m_\mu E_2 - 2m_\mu E_3 + 2 E_2 E_3 (1-\cos \theta)
\end{align*}

\begin{align*}
   \dd[3]{k_2} \dd[3]{k_3} = (4\pi) (2\pi) E_2^2 \dd{E_2} E_3^2 \dd{E_3} \dd{\cos \theta}
\end{align*}

\begin{align*}
   \delta(\dots - 2 E_2 E_3 \cos \theta) = \frac{1}{2E_2 E_3} \delta(- \cos \theta)
\end{align*}

\begin{align*}
   \dd{\Gamma} = \frac{G_F^2}{2 \pi^3} \dd{E_3} \dd{E_3} E_3 (m_\mu^2 - 2 m_\mu E_3)
\end{align*}

\begin{align*}
   \frac{1}{2} m_\mu - E_2 \leq E_3 \leq \frac{1}{2} m_\mu \\
   0 \leq E_2 \leq \frac{1}{2} m_\mu
\end{align*}

Carrying the integration out
\begin{align*}
   \frac{\dd{\Gamma}}{\dd{E_2}} &= \frac{m_\mu G_F^2}{2\pi^3} \int_{\frac{1}{2}m_\mu - E_2}^{\frac{1}{2}m_\mu} \dd{E_3} E_3 (m_\mu - 2 E_3) \\
                                &= \frac{G_F^2}{2\pi^3} m_\mu^2 E_2^2 \left( 3 - \frac{4E_2}{m_\mu} \right)
\end{align*}

Finally
\begin{align*}
   \Gamma = \frac{1}{\tau_\mu} = \int^{\frac{1}{2}m_\mu}_0 \dd{E_2} \frac{\dd{\Gamma}}{\dd{E_2}} = \frac{G_F^2 m_\mu^5}{192 \pi^3}
\end{align*}
If $m_e \neq 0$
\begin{align*}
   \Gamma = \frac{1}{\tau_\mu} = \int^{\frac{1}{2}m_\mu}_0 \dd{E_2} \frac{\dd{\Gamma}}{\dd{E_2}} = \frac{G_F^2 m_\mu^5}{192 \pi^3} \left[ 1 - 8r^2 + 8r^6 - r^8 - 24 r^4 \ln{r} \right]
\end{align*}
with $r = m_e / m_\mu$. $r$ enters in boundary of phase space integration.
%%%%%%%%%%%%%%%%%%%%%%%%%%%%%%%%%%%%%%%%%%%%%%%%%%%%%%%%%%%%%%%%%
% Lecture date: 19-12-16
%%%%%%%%%%%%%%%%%%%%%%%%%%%%%%%%%%%%%%%%%%%%%%%%%%%%%%%%%%%%%%%%%

If one tries to measure the decay width precisely, the $1$-loop electroweak correction must also be included.
%TODO: diagram
\begin{align*}
   \includegraphics[width=0.5\linewidth]{muonDecay/1.eps}
\end{align*}

Neutrino can also be considered in the calculation and one finds an upper bound for neutrino mass $m_{\nu_\mu} \leq \SI{1}{\kilo \eV}$.

Mass dimension of $\Gamma$ is $+1$. Using dimensional argument, since $[G_F] = -2$, the decay width $\Gamma \sim G_F^2 m_\mu^5$. Grand Unified Theory postulates the decay of proton, $p \rightarrow e^+ \pi^0$. Following $M_{X,Y} \sim \SI{10e16}{\giga \eV}$, we estimate the decay width of proton $\Gamma \sim \frac{g_{\mathbf{SU}(5)}^4}{M_X^4} M^5_{p}$.
\begin{align*}
   \includegraphics[width=0.5\linewidth]{muonDecay/2.eps}
\end{align*}

\paragraph{Polarized Muon Decay}
In the above computation, the muon polarizations are summed over. Assume the polarization is fixed
% TODO: diagram. helicity and angular momentum conservation

Also the spin sum relation enables us the computation. How to compute decay of polarized particle then? Recall that $u$, $\bar{u}$, $v$ and $\bar{v}$ are independent solutions to Dirac equation.
\begin{align}
   u^{(s)} = \sqrt{E+m} \begin{pmatrix} \chi^{(s)} \\ \frac{\pmb{\sigma} \cdot \pmb{p}}{E+m} \chi^{(s)} \end{pmatrix}
\end{align}
To get chiral-left part of $u^{(s)}$ $P_L u^{(s)}(\pmb{p})$. Can also define two projection operators
\begin{align}
   \Lambda_+ &= \frac{1}{2m} (\slashed{p} + m) \\
   \Lambda_- &= \frac{1}{2m} (-\slashed{p} + m)
\end{align}

They are indeed projection operators
\begin{align*}
   \Lambda_+ + \Lambda_- &= \id_4 \\
   (\Lambda_+)^2 &= \frac{1}{4m^2} \left( \slashed{p} \slashed{p} + 2m \slashed{p} + m^2 \right) = \frac{1}{4m^2} 2m (\slashed{p} + m) = \Lambda_+ \\
   (\Lambda_-)^2 &= \Lambda_-
\end{align*}

Let $\sum_{r=1}^{4} a_r u^{(r)}$ be an arbitrary spinor
\begin{align*}
   \Lambda_+ &= \sum_{r=1}^{4} a_r u^{(r)} \\
             &= \sum_{r=1}^{4} a_r \left( \sum_{s=1}^{2} \frac{u^{(s)}\bar{u}^{(s)}}{2m} \right) u^{(r)}
   \shortintertext{use $\bar{u}^{(s)} u^{(r)} = 2m \delta^{rs}$}
             &= \sum_{r=1}^2 a_r u^{(r)}
\end{align*}
arbitrary $E > 0$ spinor. Thus $\Lambda_+$ projects to $E > 0$ states and $\Lambda_-$ projects $E<0$ states.

Particle at rest, spin $\pmb{s}$ and $|\pmb{s}|=1$. Write a four-vector $s^\mu = (0, \pmb{s})$. At rest $p^\mu = (m, 0)$ and $(s\cdot p) = 0$. 

Starting from the $s^\mu$, can compute $s'^\mu$ in any frame by a Lorentz boost.
\begin{align*}
   s^0 &= \frac{\pmb{p} \cdot \pmb{\xi}}{m} \\
   s^i &= \xi^i + \frac{(\pmb{p} \cdot \pmb{\xi})}{m(m+E)}p^i
\end{align*}
$\pmb{\xi}$ denotes the direction of spin and $|\pmb{\xi}| = 1$

Compute $s \cdot p$ in new frame
\begin{align*}
   s \cdot p &= \frac{\pmb{p} \cdot \pmb{\xi}}{m} E - \pmb{p} \cdot \pmb{\xi} \\
             &= \frac{\pmb{p}\cdot \pmb{\xi} \pmb{p}^2}{m (m+E)} \\
             &= \pmb{p} \cdot \pmb{\xi} \left[ \frac{E}{m} - 1 - \frac{\pmb{p}^2}{m(m+E)} \right] \\
             &=\frac{\pmb{p} \cdot \pmb{\xi}}{ m(m+E)}  \left[ E (m+E) - m(m+E) - \pmb{p}^2 \right] \\
             &= 0
\end{align*} 

Define two projection operators
\begin{align*}
   \Sigma_{\pm} &= \frac{1}{2} \left( \id_4 \pm \gamma^5 \slashed{s} \right) \\
   \Sigma_+ + \Sigma_- &= \id_4 \\
   (\Sigma_-)^2 &= \frac{1}{4} (\id - \gamma^5 \slashed{s}) (\id - \gamma^5 \slashed{s}) \\
                &= \frac{1}{4} \left[ \id - 2 \gamma^5 \slashed{s} + \gamma^5 \slashed{s} \gamma^5 \slashed{s} \right]  \\
                &= \frac{1}{4} [ 2 \id - 2 \gamma^5 \slashed{s}] \\
                &= \Sigma_- \\
   (\Sigma_+)^2 &= \Sigma_+
\end{align*}

What do $\Sigma_\pm$ project out? In rest frame $\Sigma_- = \frac{1}{2} (\id - \gamma^5 \gamma^i s^i)$. Choose $\pmb{s} = \pmb{e}_3 = \pmb{e}_z$. 
\begin{align*}
   \Sigma_- &= \frac{1}{2} (\id - \gamma^5 \gamma^3) \\
            &= \frac{1}{2} \left[ \begin{pmatrix} \id_2 & 0 \\ 0 & \id_2\end{pmatrix}  - \begin{pmatrix} 0 &\id \\ \id & 0 \end{pmatrix} \begin{pmatrix} 0 & \sigma^3 \\ -\sigma^3 & 0 \end{pmatrix}\right] \\
            &= \frac{1}{2} \begin{pmatrix} \id_2 - \sigma^3 & 0 \\ 0 & \id_2 + \sigma^3 \end{pmatrix}
\end{align*}
so project out helicity states.

\begin{align}
   u^k(\pmb{p},s) \bar{u}_i ( \pmb{p}, s) = \frac{1}{2} \left[ (\slashed{p}+ m) (\id - \gamma^5 \slashed{s}) \right]_i^k
\end{align}
or in computation
\begin{align}
   (\slashed{p} + m ) \mapsto \frac{1}{2} (\slashed{p} + m_\mu) (\id - \gamma^5 \slashed{s}_\mu)
\end{align}

Turns out we can skip computation. Replace $p_\alpha \mapsto p_\alpha- m s_\alpha$ in final answer
\begin{align*}
   &\tr[\dots P_R (\slashed{p} + m_\mu) (\id -\gamma^5 \slashed{s}) \gamma_\alpha P_R \dots ] \\
   &= \tr \left[ \dots P_R (\slashed{p} - m_\mu \gamma^5 \slashed{s}) \gamma_\alpha P_R \dots \right] \\
   &= \tr \left[ \dots P_R (\slashed{p} - m \slashed{s}) \gamma_\alpha P_R \right]
\end{align*}
with
\begin{align*}
   \pmb{\xi} \cdot \frac{\pmb{p}_e}{|\pmb{p}_e|} = \cos \theta
\end{align*}
In the end
\begin{align}
   \frac{\dd{\Gamma}}{\Gamma} = \frac{1}{2} \left(1-\frac{1}{3}\cos \theta \right) \dd{\cos \theta}
\end{align}

\paragraph{Nuclear $\beta$-decay}
${}^{14}O \rightarrow {}^{14}N^* e^+ \nu_e$ $\beta^+$-emitter
$n \rightarrow p e^- \bar{\nu}_e$

Non-relativistic limit of Pauli-Dirac spinor
\begin{align*}
   u^{(s)}(\pmb{p}) &= \sqrt{E+m} \begin{pmatrix} \chi^{(s)} \\ \frac{\pmb{\sigma}\cdot \pmb{p}}{E + m} \chi^{(s)}\end{pmatrix} \\
   &\rightarrow \sqrt{2m} \begin{pmatrix} \chi^{(s)} \\ 0 \end{pmatrix}\\
   \bar{u}^{(s)} &= \sqrt{2m} \begin{pmatrix} x^{(s)} & 0\end{pmatrix}
\end{align*}

\begin{align*}
   &\bar{\psi}_n \gamma_\mu \frac{1}{2} \left( \id - \gamma^5 \right) \psi_p (x)
   \shortintertext{$\gamma^5$ matrice has no effect and vanishes}
   &= \frac{1}{2} \bar{\psi}_n (x) \gamma_\mu \psi_p (x) \\
   &= \frac{1}{2} \psi_n (x) \gamma^0 \gamma_\mu \psi_p (x) \\ 
   &\rightarrow  \frac{1}{2} \psi_n^\dagger \psi_p
\end{align*}

$\mu \rightarrow i=1,2,3$
\begin{align*}
   \gamma^0 \gamma^i &= \begin{pmatrix} \id_2 & 0 \\ 0 & -\id_2 \end{pmatrix} \begin{pmatrix} 0 & \sigma_i \\ -\sigma_i & 0 \end{pmatrix}\\
   &= \begin{pmatrix} 0 & \sigma_i \\ \sigma_i & 0 \end{pmatrix}
\end{align*}
It is off-diagonal.

Thus in the non-relativistic limit
\begin{align*}
   \psi_n&^\dagger \gamma^0 \gamma^i \psi_p \rightarrow 0
\end{align*}


\printbibliography
\end{document}
